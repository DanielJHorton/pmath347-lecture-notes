\documentclass[11pt]{article}
\usepackage{geometry}
\geometry{
    a4paper,
    total={170mm,257mm},
    left=15mm,top=25mm
}
\usepackage{graphicx}
\usepackage{tikz}
\usepackage{amsmath,amsfonts,amssymb,amsthm}
\usepackage{hyperref}
\usepackage{theoremref}
\usepackage{fancyhdr}
\usepackage{dhortcmds}
\usepackage{dhortthms}

% Pagestyle Setup
\pagestyle{fancy}
\setlength{\headheight}{13.6pt}

% Headers and Footers
\renewcommand{\headrulewidth}{0.4pt}% default is 0.4pt
\renewcommand{\footrulewidth}{0.4pt}% default is 0pt
\lhead{\rightmark} % \rightmark gives the current section
\rhead{Instructor: Yu-Ru Liu}
\lfoot{PMATH 347 - Lecture Notes}
\rfoot{Daniel Horton - Fall 2023}

\begin{document}

% Title Page
\thispagestyle{empty}
\title{PMATH 347 Groups and Rings - Lecture Notes}
\author{Instructor: Yu-Ru Liu \\ \LaTeX'd by Daniel Horton}
\date{University of Waterloo - Fall 2023}
\maketitle
\pagebreak

% Table of Contents
\tableofcontents
\pagebreak

\section{Groups}

\subsection{Basic Properties}

\begin{definition}[Groups]
    Let $G$ be a set and $*:G\times G\to G$. The pair $(G,*)$ is a \ul{group} if it satisfies the following:
    \begin{enumerate}
        \item\emph{Closure}: If $a,b\in G$, then $a*b\in G$.
        \item\emph{Associativity}: If $a,b,c\in G$ then $a*(b*c)=(a*b)*c$.
        \item\emph{Identity}: There exists $e\in G$ such that $a*e=a=e*a$ for all $a\in G$. In this case we call this element $e$ an \ul{identity} of $G$ and it must be unique (see \thref{prop1.1}).
        \item\emph{Inverse}: For all $a\in G$, there exists some $b\in G$ such that $a*b=e=b*a$. We call $b$ the \ul{inverse} of $a$ and it is also unique (see \thref{prop1.1}).
    \end{enumerate}
\end{definition}

\begin{notation}
    Normally in an abstract group we denote the group operation by multiplication (i.e. $a\cdot b$) or omit it entirely (i.e. $ab$).
\end{notation}

\begin{definition}[Abelian Groups]
    A group $G$ is \ul{abelian} if $a*b=b*a$ for all $a,b\in G$.
\end{definition}

\begin{example}
    The pairs $(\Z,+),(\Q,+),(\R,+)$ and $(\C,+)$ are all abelian groups but $(\Q,\cdot),(\R,\cdot)$ and $(\C,\cdot)$ are not since they do not contain inverses for 0.
\end{example}

\begin{proposition}[Uniqueness of the Identity and Inverses]\thlabel{prop1.1}
    Let $G$ be a group with $a\in G$. There exist unique elements $e,a^{-1}\in G$ satisfying the properties of the identity and the inverse of $a$ respectively.
\end{proposition}

\begin{proof}\,
    \begin{itemize}
        \item Let $e_1,e_2$ be two identities elements of $G$. Then $e_1=e_1*e_2=e_2$ and so they are the same.
        \item Let $b_1,b_2$ be inverses of $a$ in $G$. Then
        \[b_1=b_1*e=b_1*(a*b_2)=(b_1*a)*b_2=e*b_2=b_2\]
        and so they are the same as well.
    \end{itemize}
\end{proof}

\begin{definition}[Units]
    Let $G$ be a group and $S\subseteq G$. An element $s\in S$ is a \ul{unit of $S$} if $s^{-1}\in S$ as well. We denote the set of all units of $S$ by $S^*:=\{s\in S:\exists s^{-1}\in S,s*s^{-1}=e\}$.
\end{definition}

\begin{definition}[The General Linear Group]
    For a field $\F$, we define the set $$\GL_n(\F):=\{M\in M_n(\F):\det(M)\neq0\}.$$ When combined with matrix multiplication this set forms a group called the \ul{general linear group} (see \thref{prop1.2}).
\end{definition}

\begin{proposition}\thlabel{prop1.2}
    $(\GL_n(\F),\cdot)$ is a group for all fields $\F$.
\end{proposition}

\proof\,
\begin{enumerate}
    \item\emph{Closure:} Note that if $A,B\in\GL_n$ then $\det(AB)=\det(A)\det(B)\neq0$ since $\det(A)\neq0$ and $\det(B)\neq0$.
    \item\emph{Associativity:} This follows from the associativity of matrix multiplication.
    \item\emph{Identity:} Clearly $I_n\in\GL_n(\F)$ since $\det(I_n)=1\neq0$.
    \item\emph{Inverse:} Since $\det(A)\neq0$ for all $A\in\GL_n(\F)$, we know that there exists a matrix $A^{-1}$ such that $A^{-1}A=I=AA^{-1}$. Moreover, $\det(A^{-1})=\frac{1}{\det(A)}\neq0$ since $\det(A)\neq0$ so $A^{-1}\in\GL_n(\F)$ as well.
\end{enumerate}

\begin{note}
    $\GL_n(\F)$ is only abelian in the trivial case when $n=1$.
\end{note}

\begin{notation}
    When dealing with multiple groups we use subscripts to distinguish their respective group operations and identity elements (i.e. $e_G*_Ga=a$).
\end{notation}

\begin{definition}[Direct Product Groups]
    Let $(G,*_G),(H,*_H)$ be groups. Their direct product group is $(G\times H,*)$ where $*:(G\times H)\times(G\times H)\to G\times H$ is defined by
    \[(g_1,h_1)*(g_2,h_2)=(g_1*_Gg_2,h_1*_Hh_2).\]
    This group has the identity $e=(e_G,e_H)$ and the inverse of $(g,h)\in G\times H$ is $(g^{-1},h^{-1})$.
\end{definition}

\begin{proposition}[Basic Group Identities]
    Let $G$ be a group and $g,g\in G$.
    \begin{enumerate}
        \item $(g^{-1})^{-1}=g$
        \item $(gh)^{-1}=h^{-1}g^{-1}$
        \item $g^n\cdot g^m=g^{n+m}$ for all $n,m\in\Z$
        \item $(g^n)^m=g^{nm}$ for all $n,m\in\Z$
    \end{enumerate}
\end{proposition}

\begin{proof}
    The proof is left as an exercise.
\end{proof}

\begin{note}
    In general $(gh)^n\neq g^nh^n$. This requires $g$ and $h$ to commute.
\end{note}

\begin{proposition}[Algebraic Manipulation of Groups]
    Let $G$ be a group and $f,g,h\in G$. Then:
    \begin{enumerate}
        \item They satisfy left and right cancellation, namely:
        \begin{enumerate}
            \item If $gh=gf$ then $h=f$.
            \item If $hg=fg$ then $h=f$.
        \end{enumerate}
        \item The equations $gx=h$ and $yg=h$ have unique solutions for $x,y\in G$.
    \end{enumerate}
\end{proposition}


\begin{proof}\,
\begin{enumerate}
    \item\begin{enumerate}
        \item By multiplying by $g^{-1}$ on the left we get:
        \[gh=gf\implies g^{-1}gh=g^{-1}gf\implies h=f\]
        \item Same as before but multiply by $g^{-1}$ on the right.
    \end{enumerate}
    \item Let $x=g^{-1}h$ and $y=hg^{-1}$. Clearly these are valid solutions. Now assume that $u,v$ are additional solutions such that $gu=h$ and $vg=h$. This gives $gu=gx$ and $vg=yg$ and so by (1), $u=x$ and $v=y$. 
\end{enumerate}
\end{proof}

\subsection{Symmetric Groups and Cycles}

\begin{definition}[Permutations]
    Given a non-empty set $L$, a \ul{permutation of $L$} is a bijection from $L$ to $L$.
\end{definition}

\begin{definition}[Symmetric Groups]
     Given a non-empty set $L$, The set of all permutations of $L$ is denoted by $S_L$ and forms a group with function composition called the \ul{symmetric group over $L$}.
\end{definition}

\begin{note}
    The identity element $\epsilon\in S_L$ is the identity transformation $\epsilon:L\to L$ with $\epsilon(x)=x$
\end{note}

\begin{notation}
    For $n\in\N$ let $S_n$ denote the symmetric group over the set $\{1,\cdots,n\}$.
\end{notation}

\begin{proposition}
    For all $n\in\N$, $|S_n|=n!$.
\end{proposition}

\begin{proof}
    This follows from that fact that there are $n!$ ways to permute $n$ objects.
\end{proof}

\begin{definition}[Cycle Notation]
    We can define elements $\sigma\in S_n$ by the cycles they form with the elements of $\{1,2,\cdots,n\}$.
\end{definition}

\begin{example}
    Let $\sigma,\tau\in S_4$ with 
    \[\sigma(x)=\begin{cases}2&x=1\\1&x=2\\4&x=3\\3&x=4\end{cases}\text{ and }\tau(x))=x+1\mod{4}.\]
    Then in cycle notation we have:
    \begin{align*}
        \sigma&=(1\,2)(3\,4) \\
        \sigma^{-1}&=(2\,1)(4\,3) \\
        \tau&=(1\,2\,3\,4) \\
        \tau^{-1}&=(4\,3\,2\,1) \\
        \sigma\tau&=(2\,4) \\
        \tau\sigma&=(1\,3)
    \end{align*}
\end{example}

\begin{theorem}[Cycle Decomposition Theorem]\thlabel{thm1.6}
    If $\sigma\in S_n$ with $\sigma\neq\epsilon$, then $\sigma$ is a product of (one or more) disjoint cycles of length at least 2 and this factorization is unique up to the order of the factors.
\end{theorem}

\begin{proof}
    The proof of \thref{thm1.6} is left as an exercise (difficult).
\end{proof}

\subsection{Cayley Tables}

\begin{definition}[Cayley Tables]
    Given a group $G$ and $x,y\in G$, if we write the product $xy$ as the entry of a table in the row corresponding to $x$ and column corresponding to $y$ then such a table is called a \ul{Cayley table}.
\end{definition}

\begin{note}
    By cancellation the entries in each row (or in each column) of a Cayley table are all distinct.
\end{note}

\begin{example}
    Consider the groups $(\Z_2,+)$ and $(\Z^*,\cdot)$ where $\Z^*=\{-1,1\}$. These groups have the following Cayley tables:
    \[\begin{tabular}{c|cc}
        $\Z_2$&[0]&[1] \\\hline
        [0]&[0]&[1] \\
        
        [1]&[1]&[0]
    \end{tabular}\qquad\begin{tabular}{c|cc}
        $\Z^*$&-1&1\\\hline
        -1&1&-1\\

        1&-1&1
    \end{tabular}\]
\end{example}

\begin{definition}[Cyclic Groups]
    The \ul{cyclic group of order $n$} is defined by
    \[C_n:=(\{1,a,a^2,\cdots,a^{n-1}\},\cdot)\]
    with $a^n=1$ and $1,a,a^2,\cdots,a^{n-1}$ all distinct.
\end{definition}

\begin{example}
    The Cayley table for $C_n$ is:
    \[\begin{tabular}{c|ccccc}
        $C_n$ & 1 & $a$ & $a^2$ & $\cdots$ & $a_{n-1}$ \\ \hline
        1 & 1 & $a$ & $a^2$ & $\cdots$ & $a_{n-1}$ \\

        $a$ & $a$ & $a^2$ & $a^3$ & $\cdots$ & 1 \\

        $a^2$ & $a^2$ & $a^3$ & $a^4$ & $\cdots$ & $a$ \\
        
        $\vdots$ & $\vdots$ & $\vdots$ & $\vdots$ & $\ddots$ & $\vdots$ \\
        $a_{n-1}$ & $a_{n-1}$ & 1 & $a$ & $\cdots$ & $a_{n-2}$
    \end{tabular}\]
\end{example}

\begin{definition}[Isomorphic Groups]\thlabel{def1.12}
    We say that two groups $G,H$ are \ul{isomorphic} if their Cayley tables are the same (up to the naming and ordering of elements), in which case we write $G\iso H$.
\end{definition}

\begin{proposition}[Classification of Groups up to Order 4]
    Let $G$ be a group.
    \begin{enumerate}
        \item If $|G|=1$, then $G\iso\{1\}$.
        \item If $|G|=2$, then $G\iso C_2$.
        \item If $|G|=3$, then $G\iso C_3$.
        \item If $|G|=4$, then $G\iso C_4$ or $G\iso K_4=C_2\times C_2$.
    \end{enumerate}
\end{proposition}

\begin{proof}\,
\begin{enumerate}
    \item If $|G|=1$ then trivially $G=\{1\}$ up to isomorphism.
    \item If $|G|=2$, then $G=\{1,g\}$ with $g\neq 1$. We must have a $g^{-1}$ in $G$ with $g^{-1}\neq 1$, so $g^{-1}=g$. This gives the Cayley table for $C_2$.
    \item If $|G|=3$, then $G=\{1,g,h\}$ with $g\neq 1$, $h\neq1$ and $g\neq h$. By cancellation, $gh\neq g$ and $gh\neq h$. Thus $gh=1$. Similarly, $hg=1$. Since all entries on the same row of a Cayley table need to be distinct, this then gives $g^2=h$ and $h^2=g$. By observation of the resulting Cayley table, if we let $g=a$ and $h=a^2$ then we get $G\iso C_3$.
    \item See assignment 1.
\end{enumerate}
\end{proof}

\begin{definition}[The Klein 4-Group]
    The group $K_4\iso C_2\times C_2$ is called the \ul{Klein 4-group}.
\end{definition}

\section{Subgroups}

\subsection{Subgroups}

\begin{definition}[Subgroups]
    Let $G$ be a group and $H\subseteq G$ be a subset of $G$. If $H$ is itself a group (using $G$'s group operation) then we say that $H$ is a \ul{subgroup} of $G$.
\end{definition}

\begin{theorem}[The Subgroup Test]\thlabel{subgrouptest}
    Let $G$ be a group and $H\subseteq G$. $H$ is a subgroup if and only if the following are all true.
    \begin{enumerate}
        \item If $h_1,h_2\in H$ then $h_1h_2\in H$.
        \item $1_G\in H$.
        \item If $h\in H$, then $h^{-1}\in H$.
    \end{enumerate}
\end{theorem}

\begin{proof}\,
    \begin{itemize}
        \item[$\implies$] This follows directly from the properties of a group.
        \item[$\impliedby$] By assumption the closure, identity and inverse properties of a group hold for $H$. As for associativity, since $G$ was a group the group operation was associative over $G$, and so it must be associative over $H$ as well. Hence $H$ is a group.
    \end{itemize}
\end{proof}

\begin{remark}
    Essentially, what the subgroup test is saying is that when determining if a subset is or isn't a subgroup we may ignore the associativity group axiom.
\end{remark}

\begin{definition}[The Special Linear Group]
    The group $\SL_n(\F)=\{M\in M_n(\F):\det(M)=1\}$ is called the \ul{special linear group} and it is a subgroup of $\GL_n(\F)$.
\end{definition}

\begin{definition}[The Center of a Group]
    The \ul{center of a group} $G$ is the set
    \[Z(G):=\{z\in G:\forall x\in G,zx=xz\}\]
    and it is an abelian subgroup of $G$ (see \thref{prop2.2}).
\end{definition}

\begin{proposition}\thlabel{prop2.2}
    For all groups $G$, $Z(G)$ is an abelian subgroup of $G$.
\end{proposition}

\begin{proof}
    Note that $Z(G)$ is by definition abelian, so we just need to show that it is a subgroup. We proceed by using the subgroup test (\thref{subgrouptest}).
    \begin{enumerate}
        \item Let $y,z\in Z(G)$. By associativity and the definition of the center we have:
        \[(yz)g=y(zg)=y(gz)=(yg)z=(gy)z=g(yz)\]
        So $yz\in Z(G)$.
        \item Clearly $1_G\in Z(G)$ since by definition $1g=g=g1$.
        \item Let $z\in Z(G)$ and $g\in G$. we have:
        \[zg=gz\iff z^{-1}(gz)z^{-1}=z^{-1}(zg)z^{-1}\iff z^{-1}g=gz^{-1}\]
        So $z^{-1}\in Z(G)$ as well.
    \end{enumerate}
\end{proof}

\begin{proposition}\thlabel{prop2.3}
    Let $H,K$ be subgroups of $G$. Then $H\cap K$ is a subgroup of $G$ as well.
\end{proposition}

\begin{proof}
    This follows easily from the subgroup test (\thref{subgrouptest}).
\end{proof}

\begin{proposition}[The Finite Subgroup Test]
    If $H$ is a finite non-empty subset of $G$, then $H$ is a subgroup of $G$ if and only if $H$ is closed under $G$'s group operation.
\end{proposition}

\begin{proof}\,
    \begin{itemize}
        \item[$\implies$] This follows from the closure of a subgroup.
        \item[$\impliedby$] If $H=\emptyset$ then the statement is vacuously true, so assume there exists some $h\in H$. Since $H$ is closed under the group operations, $\{h,h^2,h^3,\cdots\}\subseteq H$. Since $h$ is finite these elements are not all distinct and there exists some $n,m\in N$ such that $h^n=h^{n+m}$. By cancellation this implies that $h^m=1$ so $1\in H$. Furthermore, $h\cdot h^{m-1}=h^m=1$ so $h^{-1}=h^{m-1}\in H$ as well. Thus by the subgroup test (\thref{subgrouptest}) $H$ is a subgroup of $G$.
    \end{itemize}
\end{proof}

\subsection{Alternating Groups}

Recall that by the cycle decomposition theorem (\thref{thm1.6}) for $\sigma\in S_n$, with $\sigma\neq\epsilon$, $\sigma$ can be decomposed uniquely (up to the ordering of its factors) as disjoint cycles of length at least 2.

\begin{definition}[Transpositions]
    A transposition $\sigma\in S_n$ is a cycle of length 2.
\end{definition}

\begin{remark}
    All larger cycles can be decomposed into transpositions (e.g. $(1\,2\,4\,5)=(1\,2)(2\,4)(4\,5)\in S_5$) however this factorization is not unique.
\end{remark}

\begin{theorem}[Parity Theorem]\thlabel{paritythm}
    If a permutation $\sigma$ has two factorizations
    \[\sigma=\gamma_1\cdots\gamma_r=\mu_1\cdots\mu_s\]
    where each $\gamma_i$ and $\mu_j$ is a transposition, then $r\equiv s\pmod{2}$.
\end{theorem}

\proof The proof of the parity theorem is left as a (difficult) exercise.

\begin{definition}[Even and Odd Permutations]
    A permutation $\sigma$ is \ul{even or odd} if it can be written as a product of an even or odd number of transpositions. By the parity theorem (\thref{paritythm}) a permutation is either even or odd but not both.
\end{definition}

\begin{definition}[Alternating Groups]
    For $n\geq2$, the \ul{alternating group} $A_n$ is the set of all even permutations of $S_n$.
\end{definition}

\begin{note}\,
    \begin{enumerate}
        \item It follows from the next proposition (\thref{prop2.6}) and the subgroup test (\thref{subgrouptest}) that this set is a group.
        \item By construction, $A_n$ is a subgroup of $S_n$.
    \end{enumerate}
\end{note}

\begin{proposition}[Properties of Alternating Groups]\thlabel{prop2.6}
    For $n\geq2$ let $A_n$ denote the set of all even permutations of $S_n$.
    \begin{enumerate}
        \item $\epsilon\in A_n$.
        \item If $\sigma,\tau\in A_n$, then $\sigma\tau\in A_n$ and $\sigma^{-1}\in A_n$.
        \item $|A_n|=\frac{1}{2}n!=\frac{1}{2}|S_n|$.
    \end{enumerate}
\end{proposition}

\begin{proof}\,
\begin{enumerate}
    \item We can write $\epsilon=(1\,2)(1\,2)$ so $\epsilon$ is even and thus $\epsilon\in A_n$.
    \item If $\sigma,\tau\in A_n$ we have
    \[\sigma=\sigma_1\cdots\sigma_{2r},\tau=\tau_1\cdots\tau_{2s}\]
    where $r,s\in\N$. This gives
    \[\sigma\tau=\sigma_1\cdots\sigma_{2r}\tau_1\cdots\tau_{2s}\]
    so clearly $\sigma\tau\in A_n$ as well.

    As for $\sigma^{-1}$, we note that since each $\sigma_i$ is a transposition, $\sigma_i^{-1}=\sigma_i$. This gives
    \[\sigma^{-1}=(\sigma_1\cdots\sigma_{2r})^{-1}=\sigma_{2r}^{-1}\cdots\sigma_1^{-1}=\sigma_{2r}\cdots\sigma_1\]
    so $\sigma^{-1}\in A_n$ as well.
    \item Let $O_n$ denote the set of odd permutations in $S_n$. By the parity theorem we have $S_n=A_n\cup O_n$ and $A_n\cap O_n=\emptyset$. Since $|S_n|=n!$, it now suffices to show that $|O_n|=|A_n|$. Let $\gamma=(1\,2)$ and define $f:A_n\to O_n$ by $f(\sigma)=\gamma\sigma$. Since $\sigma\in A_n$, $\gamma\sigma\in O_n$ and so $f$ is well defined. Also, if $\gamma\sigma_1=\gamma\sigma_2$ then $\sigma_1=\sigma_2$ so $f$ is 1 to 1. Finally, observe that
    \[f(f(\sigma))=(1\,2)(1\,2)\sigma=\sigma\]
    so $f^{-1}=f$ and thus $f$ is a bijection from $A_n$ to $O_n$. We conclude $|A_n|=|O_n|$ and so $|A_n|=\frac{1}{2}n!$.
\end{enumerate}
\end{proof}

\subsection{Orders of Elements}

\begin{definition}[Cyclic Subgroups]
    Let $G$ is a group and $g\in G$. We define
    \[\langle g\rangle=\{g^n:n\in\Z\}=\{\cdots,g^{-2},g^{-1},1,g,g^2,\cdots\}.\]
    This set is called the cyclic subgroup generated by $g$ (see \thref{prop2.7}).
\end{definition}

\begin{proposition}\thlabel{prop2.7}
    If $G$ is a group then for all $g\in G$, $\langle g\rangle$ is a subgroup of $G$.
\end{proposition}

\begin{proof}
    This follows from the subgroup test (\thref{subgrouptest}).
\end{proof}

\begin{definition}[Orders of Elements]
    Let $G$ be a group and $g\in G$. If $n$ is the smallest positive integer such that $g^n=1$, then we say \ul{the order} of $g$ is $n$ (denoted $o(g)=n$). If no such $n$ exists then we say that the order of $g$ is infinite ($o(g)=\infty$).
\end{definition}

\begin{proposition}[Properties of Elements with Finite Order]\thlabel{prop2.8}
    Let $G$ be a group and $g\in G$ with $o(g)=n\in\N$. For $k\in\Z$, we have:
    \begin{enumerate}
        \item $g^k=1\iff n|k$
        \item $g^k=g^m\iff k\equiv m\pmod{n}$
        \item $\langle g\rangle=\{1,g,g^2,\cdots,g^{n-1}\}$
    \end{enumerate}
\end{proposition}

\begin{proof}\,
\begin{enumerate}
    \item\,
    \begin{itemize}
        \item [$\impliedby$] If $n|k$ then $k=nq$ for some $q\in\Z$. Thus \[g^k=g^{nq}=(g^n)^q=1^q=1.\]
        \item[$\implies$] By the division algorithm we can write $k=nq+r$ with $q,r\in\Z$ and $0\leq r<n$. Since $g^k=1$ and $g^n=1$ we have,
        \[g^r=g^{k-nq}=g^k(g^n)^{-q}=1(1)^{-q}=1.\]
        Since $0\leq r<n$ and $o(g)=n$, it follows that $r=0$ and hence $n|k$.
    \end{itemize}
    \item Note that $g^k=g^m\iff g^{k-m}=1$. By (1) this follows if and only if we have $n|(k-m)$.

    \item It follows from (2) that $1,g,\cdots,g^{n-1}$ are all distinct. Clearly, we have $\{1,g,\cdots,g^{n-1}\}\subseteq\langle g\rangle$. To prove that other inclusion, let $x=g^k\in\langle g\rangle$ for some $k\in\Z$. Write $k=nq+r$ with $q,r\in\Z$ and $0\leq r<n$. Then
    \[x=g^k=(g^n)^qg^r=1^qg^r=g^r\in\{1,g,\cdots,g^{n-1}\}.\]
\end{enumerate}
\end{proof}

\begin{proposition}[Properties of Elements with Infinite Order]\thlabel{prop2.9}
    Let $G$ be a group and $g\in G$ satisfying $o(g)=\infty$. For $k\in\Z$, we have:
    \begin{enumerate}
        \item $g^k=1\iff k=0$.
        \item $g^k=g^m\iff k=m$.
        \item $\langle g\rangle=\{\cdots,g^{-2},g^{-1},1,g,g^2,\cdots\}$ are all distinct.
    \end{enumerate}
\end{proposition}

\begin{proof}\,
\begin{enumerate}
    \item\,
    \begin{itemize}
        \item[$\implies$] $g^0=1$ by definition.
        \item[$\impliedby$] If $g^k=1$ for some $k\neq0$, then $g^{-k}=(g^k)^{-1}=1$ as well. Thus we can assume $k\geq1$. However, this implies that $o(g)$ is finite.
    \end{itemize}
    \item Note that $g^k=g^m\iff g^{k-m}=1$. By (1) this is equivalent to $k-m=0$, i.e. $k=m$.
    \item This follows directly from (2).
\end{enumerate}
\end{proof}

\begin{proposition}\thlabel{prop2.10}
    Let $G$ be a group and $g\in G$ with $o(g)=n\in\N$. For all $d\in\N$, $o(g^d)=\frac{n}{\gcd(n,d)}$. In particular, if $d|n$ then $\gcd(n,d)=d$ and $o(g^d)=\frac{n}{d}$.
\end{proposition}

\begin{proof}
    Let $n_1=\frac{n}{\gcd(n,d)}$ and $d_1=\frac{d}{\gcd(n,d)}$. By a result from MATH135, we have $\gcd(n_1,d_1)=1$. Note that
    \[(g^d)^{n_1}=(g^d)^{\frac{n}{\gcd(n,d)}}=(g^n)^{\frac{d}{\gcd(n,d)}}=1.\]
    Thus it remains to show that $n_1$ is the smallest such positive integer.
    
    Now suppose $(g^d)^r=1$ with $r\in\N$. Since $o(g)=n$, by theorem 2.8 we have $n|dr$. Thus there exists $q\in\Z$ such that $dr=nq$. Dividing both sides by $\gcd(n,d)$ gives
    \[d_1r=\frac{d}{\gcd(n,d)}r=\frac{n}{\gcd(n,d)}q=n_1q.\]
    Since $n_1|d_1r$ and $\gcd(n_1,d_1)=1$, by another result from MATH135 we get $n_1|r_1$, i.e. $r=n_1\ell$ for some $\ell\in\Z$.
    
    Since $r_1,n_1\in\N$ it follows that $\ell\in\N$. Since $\ell\geq1$, we get $r\geq n$.
\end{proof}

\subsection{Cyclic Groups}

\begin{proposition}
    Every cyclic group is abelian.
\end{proposition}

\begin{proof}
    Let $C_n$ be a cyclic group and $g,h\in C_n$. By the definition of a cyclic group $g=a^{k_1}$ and $h=a^{k_2}$ for some $k_1,k_2\in\Z$. Thus
    \[gh=a^{k_1}a^{k_2}=a^{k_1+k_2}=a^{k_2}a^{k_1}=hg\]
    so $C_n$ is abelian.
\end{proof}

\begin{proposition}
    Every subgroup of a cyclic group is cyclic.
\end{proposition}

\begin{proof}
    Let $G=\langle g\rangle$ be cyclic and let $H$ be a subgroup of $G$. If $H=\{1\}$ then we are done so we may assume there exists a $g^k\in H$ with $k\neq 0$. Since $H$ is group, $g^{-k}\in H$ as well so we may also assume $k\in\N$.

    Let $m$ be the smallest positive integer such that $g^m\in H$. Note that this means $\langle g^m\rangle\subseteq H$. To prove $H\subseteq\langle g^m\rangle$ we note that for all $h\in H$, $h=g^k$ for some $k\in\N$. By the division algorithm $k=mq+r$ with $0\leq r<m$. Thus
    \[g^r=g^{k-mq}=g^k(g^m)^{-q}\in H\]
    so $r=0$, $m|k$ and $H\subseteq\langle g^m\rangle$.
\end{proof}

\begin{proposition}
    Let $G=\langle g\rangle$ be cyclic with $o(g)=n\in\N$. Then $G=\langle g^k\rangle$ if and only if $\gcd(k,n)=1$.
\end{proposition}

\proof This follows directly from \thref{prop2.10} and \thref{prop2.8}.

\begin{theorem}[Fundamental Theorem of Finite Cyclic Groups]\thlabel{thm2.14} Let $G=\langle g\rangle$ be a cyclic group of order $n$. Then:
\begin{enumerate}
    \item If $H$ is a subgroup of $G$ then $H=\langle g^d\rangle$ for some $d|n$. Note it follows that $|H|\big|n$.
    \item Conversely, if $k|n$ then $\langle g^{n/k}\rangle$ is a subgroup of $G$ of order $k$.
\end{enumerate}
\end{theorem}

\begin{proof}\,
    \begin{enumerate}
    \item By theorem 2.12 $H$ is cyclic so we have $H=\langle g^m\rangle$ for some $m\in\N$. Let $d=\gcd(m,n)$, by a theorem from MATH135 there exist $x,y\in\Z$ such that $d=mx+ny$.
    \[g^d=(g^m)^x(g^n)^y=(g^m)^x1^y=(g^m)^x\in\langle g^m\rangle\]

    \item By theorem 2.10 $o(g^{n/k})=\frac{n}{\gcd(n,k)}=\frac{n}{\frac{n}{k}}=k$. Now suppose $K$ is any subgroup of $G$ of order $k$. By (1) $K=\langle g^d\rangle$ for some $d|n$. By theorem 2.8 we have $|K|=o(g^d)$. By theorem 2.10 we have $o(g^d)=\frac{n}{\gcd(n,d)}=\frac{n}{d}$. Since $k=|K|$ we conclude $k=\frac{n}{d}$. It follows that $d=\frac{n}{k}$ and $K=\langle g^{\frac{n}{k}}\rangle$.
\end{enumerate}
\end{proof}

\subsection{Non-Cyclic Groups}

\begin{definition}[Group Generators]
    Given a group $G$ and a nonempty subset $X\subseteq G$, the\\ \ul{subgroup generated by $X$} is the set
    \[\langle X\rangle=\{x_1^{k_1}x_2^{k_2}\cdots x_m^{k_m}:k_i\in\Z\}.\]
\end{definition}

\begin{note}
    $\langle X\rangle$ is clearly a group by the subgroup test.
\end{note}

\begin{definition}[Dihedral Groups]
    For $n\geq2$ the \ul{dihedral group of order $2n$} is defined by the set
    \[D_{2n}=\{1,a,\cdots,a^{n-1},b,ba,\cdots,ba^{n-1}\}\]
    where $a^n=b^2=1$ and $aba=b$. Thus
    \[D_{2n}=\langle a,b:a^n=b^2=1,aba=b\rangle.\]
\end{definition}

\begin{remark}
    In the cases $n=2$ and $n=3$ we have $D_4\iso K_4$ and $D_6\iso S_3$.
\end{remark}

\section{Normal Subgroups}

\subsection{Homomorphisms and Isomorphisms}

\begin{definition}[Group Homomorphisms]
    Let $G$ and $H$ be groups. A mapping $\alpha:G\to H$ is a \ul{group homomorphism} if
    \[\alpha(a*_Gb)=\alpha(a)*_H\alpha(b)\]
    for all $a,b\in G$.
\end{definition}

\begin{example}
    Given a field $\F$, the map $\det:\GL_n(\F)\to\F^*$ is a group homomorphism where $\F^*=\F\setminus\{0\}$.
\end{example}

\begin{proposition}[Properties of Group Homomorphisms]
    Let $\alpha:G\to H$ be a group homomorphism. Then:
    \begin{enumerate}
        \item $\alpha(1_G)=1_H$
        \item For all $g\in G,\alpha(g^{-1})=\alpha(g)^{-1}$
        \item For all $g\in G$ and $k\in\Z,\alpha(g^k)=\alpha(g)^k$
    \end{enumerate}
\end{proposition}

\begin{proof}
    The proof is left as an exercise.
\end{proof}

\begin{definition}[Group Isomorphisms]
    Let $G$ and $H$ be groups. Consider $\alpha:G\to H$. If $\alpha$ is a bijective homomorphism then it is a \ul{group isomorphism}.
\end{definition}

\begin{proposition}
    For all groups $G,H$,
    \[G\iso H\iff\text{ there exists a group isomorphism $\alpha:G\to H$}.\]
\end{proposition}

\begin{remark}
    Recall that we defined $G\iso H$ in-terms of Cayley tables in \thref{def1.12}
\end{remark}

\begin{proof}
    The proof is left as an exercise.
\end{proof}

\begin{proposition}[Properties of Group Isomorphisms]\thlabel{prop3.3} Let $G,H$ and $K$ be groups.
    \begin{enumerate}
        \item The identity map $G\to G$ is a group isomorphism.
        \item If $\sigma:G\to H$ is a group isomorphism then so is $\sigma^{-1}:H\to G$.
        \item If $\sigma:G\to H$ and $\tau:H\to K$ are both group isomorphisms then $\sigma\tau=\sigma\circ\tau:G\to K$ is also a group isomorphism.
    \end{enumerate}
\end{proposition}

\begin{proof}
    The proof is left as an exercise.
\end{proof}

\begin{example}
    $(\R,+)\iso(\R_{++},\times)$ where $\R_{++}=\{r\in\R:r>0\}$ since the function $\sigma(x)=e^x$ is a group isomorphism from $\R\to\R_{++}$.
\end{example}

\begin{example}
    $(\Q,+)\not\iso(\Q^*,\times)$ since if there existed a group isomorphism $\tau:\Q\to\Q^*$ then there would exist a $q\in\Q$ such that $\tau(q)=2$. Letting $\tau(\frac{q}{2})=a$ this gives:
    \[2=\tau(q)=\tau\left(\frac q2+\frac q2\right)=\tau\left(\frac q2\right)\tau\left(\frac q2\right)=a^2\]
    So $a\in\Q^*$ with $a^2=2$ which is a contradiction.
\end{example}

\begin{corollary}
    The existence of an isomorphism between two groups forms an equivalence relation.
\end{corollary}

\begin{proof}
    The proof is left as an exercise (hint, use part 3 of \thref{prop3.3}).
\end{proof}

\subsection{Cosets and Lagrange's Theorem}

\begin{definition}[Cosets]
    Let $H$ be a subgroup of $G$ and $a\in G$. The \ul{right coset} of $H$ generated by $a$ is the set
    \[Ha:=\{ha:h\in H\}.\]
    Similarly, the \ul{left coset} of $H$ generated by $a$ is
    \[aH:=\{ah:h\in H\}.\]
\end{definition}

\begin{proposition}[Properties of Cosets]\thlabel{prop3.4}
    Let $H$ be a subgroup of $G$, and let $a,b\in G$.
    \begin{enumerate}
        \item $Ha=Hb\iff ab^{-1}\in H$. In particular, $Ha=H\iff a\in H$.
        \item If $a\in Hb$, then $Ha=Hb$.
        \item Either $Ha=Hb$ or $Ha\cap Hb=\emptyset$. Thus the distinct right cosets of $H$ form a partition of $G$.
    \end{enumerate}
\end{proposition}

\begin{proof}\,
\begin{enumerate}
    \item If $Ha=Hb$ then $a=1\cdot a\in Ha=Hb$ and so $a=hb$ for some $h\in H$ so $ab^{-1}=h$. Conversely, if $ab^{-1}\in H$ then for all $h\in H$ we may write $ha=hab^{-1}b=h^2b\in Hb$ so $Ha\subseteq Hb$. On the other hand, if $ab^{-1}=h$ then $ba^{-1}=h^{-1}\in H$. So for all $h\in H$, $hb=hba^{-1}a=hh^{-1}a=a\in Ha$ so $Hb\subseteq Ha$ as well.

    \item If $a\in Hb$, then $a=hb\implies ab^{-1}=h$ for some $h\in H$ and so by (1), $Ha=Hb$.

    \item If $Ha\cap Hb=\emptyset$ then we are done, so we may assume there exists an $x\in Ha\cap Hb$. Since $x\in Ha$ and $x\in Hb$, by (2) we have $Ha=Hx=Hb$.
\end{enumerate}
\end{proof}

\begin{definition}[Indices of Subgroups]
    Let $G$ be a group and $H$ be a subgroup of $G$. The \ul{index} of $H$ in $G$ denoted $[G:H]$ is the number of distinct right (or left) cosets of $H$ in $G$.
\end{definition}

\begin{theorem}[Lagrange's Theorem]\thlabel{lagrangesthm}
    Let $H$ be a subgroup of a finite group $G$. We have $|H|\big||G|$ and
    \[[G:H]=\frac{|G|}{|H|}.\]
\end{theorem}

\begin{proof}
    This follows from part 3 of \thref{prop3.4}.
\end{proof}

\begin{corollary}\thlabel{cor3.5.1}
    If $G$ is a finite group and $g\in G$ then $o(g)\big||G|$.
\end{corollary}

\begin{proof}
     Take $H=\langle g\rangle$, notice that $|H|=o(g)$. The result follows from Lagrange's theorem (\thref{lagrangesthm}).
\end{proof}

\begin{corollary}
    If $G$ is a finite group with $|G|=n$, then for all $g\in G$, we have $g^n=1$.
\end{corollary}

\begin{proof}
    This follows from \thref{cor3.5.1} and \thref{prop2.8}.
\end{proof}

\begin{definition}[Euler's $\psi$-Function]
    \ul{Euler's $\psi$-function}, $\psi:\N\to\N$ is defined as
    \[\psi(n)=\#\{k\in\{0,\cdots,n-1\}:\gcd(k,n)=1.\]
\end{definition}

\begin{example}
    For $n\geq 2$ let $\Z_n^*$ be the set of (multiplicatively) invertible elements in $\Z_n$. Then
    \[\psi(n)=|\Z_n|.\]
    This follows from a result from MATH135.
\end{example}

\begin{theorem}[Euler's Theorem]\thlabel{eulersthm}
    If $a\in\Z$ with $\gcd(a,n)=1$ then $a^{\psi(n)}\equiv 1\pmod{n}$. When $n=p$ is prime this gives Fermat's little theorem.
\end{theorem}

\begin{proof}
    This follows from the definition of $\psi(n)$ and \thref{cor3.5.1}.
\end{proof}

\begin{proposition}
    If $G$ is a group with $|G|=p$ for some prime $p$ then $G\iso C_p$.
\end{proposition}

\begin{proof}
    Let $g\in G$ with $g\neq 1$. Then by \thref{cor3.5.1} we have $o(g)|p$. Since $g\neq 1$ and $p$ is a prime this implies $o(g)=p$. By \thref{prop2.8} we have
    \[|\langle g\rangle|=o(g)=p.\]
    It follows that $G=\langle g\rangle\iso C_p$.
\end{proof}

\begin{proposition}
    Let $H,K$ be finite subgroups of a group $G$. If $\gcd(|H|,|K|)=1$, then $H\cap K=\{1\}$.
\end{proposition}

\begin{proof}
    By \thref{prop2.3}, $H\cap K$ is a subgroup of $G$ and $H$. By Lagrange's Theorem (\thref{lagrangesthm}), we have $|H\cap K|\big||H|$ and $|H\cap K|\big||K|$. The result follows from $\gcd(|H|,|K|)=1$.
\end{proof}

\subsection{Normal Subgroups}

\begin{definition}
    Let $H$ be a subgroup of $G$. If $gH=Hg$ for all $g\in G$, we say $H$ is \ul{normal} in $G$, denoted $H\lhd G$.
\end{definition}

\begin{theorem}[The Normality Test]\thlabel{normalitytest}
    Let $H$ be a subgroup of $G$. The following are all equivalent.
    \begin{enumerate}
        \item $H\lhd G$
        \item For all $g\in G,gHg^{-1}\subseteq H$
        \item For all $g\in G, gHg^{-1}=H$
    \end{enumerate}
\end{theorem}

\begin{proof}\,
\begin{itemize}
    \item\textbf{($1\implies 2$):} Suppose $H\lhd G$ and thus $gH=Hg$. This means for all $g\in G$ and $h\in H$, there exists an $h'\in H$ such that $gh=h'g$. Now suppose $x=ghg\in gHg^{-1}$. We have $x=h'gg^{-1}=h'\in H$ so $gHg^{-1}\subseteq H$.

    \item\textbf{($2\implies 3$):} If $g\in G$ then by (2), $gHg^{-1}\subseteq H$. Taking $g^{-1}$ in place of $g$ gives $g^{-1}Hg\subseteq H$. This implies $H\subseteq gHg^{-1}$. Thus $gHg^{-1}=H$.

    \item\textbf{($3\implies 1$):} If $gHg^{-1}=H$, then $gH=Hg$.
\end{itemize}
\end{proof}

\begin{proposition}\thlabel{prop3.10}
    If $H$ is a subgroup of $G$ and $[H:G]=2$, then $H\lhd G$.
\end{proposition}

\begin{proof}
    Let $g\in G$. If $g\in H$ then $Hg=H=gH$ and we are done, so we may assume $g\not\in H$. Since $[G:H]=2$, $G=H\cup Hg$ and $H\cap Hg=\emptyset$. Thus $Hg=G\setminus H$. Similarly, $gH=G\setminus H$. Thus $Hg=gH$ for all $g\in G$ and so $H\lhd G$.
\end{proof}

\begin{definition}[Products of Subgroups]
    Let $H,K$ be subgroups of $G$. We define their product $HK$ to be the set
    \[HK:=\{hk:h\in H\text{ and }k\in k\}.\]
\end{definition}

\begin{proposition}
    Let $H$ and $K$ be subgroups of $G$. The following are all equivalent.
    \begin{enumerate}
        \item $HK$ is a subgroup of $G$.
        \item $HK=KH$
        \item $KH$ is a subgroup of $G$
    \end{enumerate}
\end{proposition}

\begin{proof} We'll prove $(1)\iff (2)$, then $(2)\iff(3)$ will follow by interchanging $H$ and $K$.
\begin{itemize}
    \item\textbf{($2\implies1$):} We have $1\cdot1\in HK$. Also if $hk\in HK$ then $(hk)^{-1}=k^{-1}h^{-1}\in KH=HK$. Additionally, for $hk,h_1k_1\in HK$, we have $kh_1\in KH=HK$, say $kh_1=h_2k_2$. It follows that $(hk)(h_1k_1)=h(kh_1)k_1=h(h_2k_2)k_1=(hh_2)(k_2k_1)\in HK$. So by the subgroup test, $HK$ is a subgroup of $G$.
    \item\textbf{($2\implies1$):} Let $kh\in KH$. Since $H$ and $K$ are subgroups of $G$, we have $h^{-1}\in H$ and $k^{-1}\in K$. Since $HK$ is also a subgroup of $G$, we have $kh=(h^{-1}k^{-1})^{-1}\in HK$. Thus we have $KH\subseteq HK$. On the other hand, if $hk\in HK$, since $HK$ is a subgroup of $G$, we have $(hk)^{-1}=k^{-1}h^{-1}\in HK$, say $k^{-1}h^{-1}=h_1k_1$. Thus $hk=k_1^{-1}h_1^{-1}\in KH$ so $HK\subseteq KH$. It follows that $HK=KH$.
\end{itemize}
\end{proof}

\begin{proposition}\thlabel{prop3.12}
    Let $H$ and $K$ be subgroups of a group $G$.
    \begin{enumerate}
        \item If $H\lhd G$ or $K\lhd G$, then $HK=KH$ is a subgroup of $G$.
        \item If $H\lhd G$ and $K\lhd H$ then $HK\lhd G$
    \end{enumerate}
\end{proposition}

\begin{proof}\,
\begin{enumerate}
    \item Suppose $H\lhd G$. Then $HK=\bigcup\limits_{k\in K}Hk=\bigcup\limits_{k\in K}kH=KH$. By lemma 3.11, $HK=KH$ is a subgroup of $G$.

    \item If $g\in G$ and $hk\in HK$, since $H\lhd G$ and $K\lhd G$ we have
    \begin{align*}
        g^{-1}(hk)g&=g^{-1}(hgg^{-1}k)g \\
        &=(g^{-1}hg)(g^{-1}kg)\in HK
    \end{align*}
    Thus $HK\lhd G$.
\end{enumerate}
\end{proof}

\begin{definition}[Normalizers]
    Let $H$ be a subgroup of $G$. The \ul{normalizer} of $H$, denoted $N_G(H)$ is defined to be
    \[N_G(H):=\{g\in G:gH=Hg\}.\]
\end{definition}

\begin{note}
    We see that $H\lhd G\iff N_G(H)=G$. We also note that in the proof of \ref{prop3.12} we did not need the full assumption $H\lhd G$. We only needed $k\in N_G(H)$ for all $k\in K$.
\end{note}

\begin{corollary}
    Let $H$ and $K$ be subgroups of $G$. If $K\subseteq N_G(H)$, then $HK=KH$ is a subgroup of $G$.
\end{corollary}

\begin{proof}
    See the note above.
\end{proof}

\begin{proposition}\thlabel{prop3.13}
    If $H\lhd G$ and $K\lhd G$ satisfy $H\cap K=\{1\}$, then $HK\iso H\times K$.
\end{proposition}

\begin{proof}
    Define $\sigma:H\times K\to HK$ by $\sigma(h,k)=hk$.
    \begin{claim}
        If $H\lhd G$ and $K\lhd G$ such that $H\cap K=\{1\}$, then $hk=kh$ for all $h\in H$ and $k\in K$.
    \end{claim}
    \begin{proof}
        Consider $x=hk(kh)^{-1}=hkh^{-1}k^{-1}$. Note that $khk^{-1}\in kHk^{-1}\in H$. Thus $x=h(kh^{-1}k^{-1})\in H$. Similarly, one can show that $x\in K$. Since $x\in H\cap K=\{1\}$, we have $hkh^{-1}k^{-1}=1$, i.e. $hk=kh$.
    \end{proof}
    \begin{claim}\thlabel{claim3.13.2}
        $\sigma$ is an isomorphism.
    \end{claim}
    \begin{proof}
        Let $(h,k),(h_1,k_1)\in H\times K$. By claim 1 we have $h_1k=kh_1$. Thus,
        \[\sigma((h,k)(h_1,k_1))=\sigma((hh_1,kk_1))=hh_1kk_1=hkh_1k_1=\sigma((h,k))\sigma((h_1,k_1))\]
        and $\sigma$ is a homomorphism. Note that by the definition of $HK$, $\sigma$ is onto. Also, if $\sigma((h,k))=\sigma((h_1,k_1))$ then we have $hk=h_1k_1$. Thus $h_1^{-1}h=k_1k^{-1}\in H\cap K=\{1\}$ implying $h_1^{-1}h=1=k_1^{-1}k$ (i.e. $h_1=h$ and $k_1=k$). Thus $\sigma$ is 1-1, and hence it is an isomorphism.
    \end{proof}
    By \thref{claim3.13.2}, $HK\iso H\times K$.
\end{proof}

\begin{corollary}
    Let $G$ be a finite group and let $H,K$ be normal subgroups of $G$ such that $H\cap K=\{1\}$ and $|H||K|=|G|$. Then $G\iso H\times K$.
\end{corollary}

\begin{proof}
    This follows from \thref{prop3.13} and the fact that $|H\times K|=|H||K|=|G|$.
\end{proof}

\section{Isomorphism Theorems}

\subsection{Quotient Groups}

\begin{definition}[Coset Multiplication]
    Let $G$ be a group with a subgroup $K$. We define multiplication on the cosets of $K$ as
    \[Ka\cdot Kb:=Kab\]
    for all $a,b\in G$. Note that we could have $Ka=Ka_1$ and $Kb=Kb_1$ for some $a_1\neq a$ and $b_1\neq b$. Thus in order for this operation to make sense, a necessary condition is
    \[Ka=Ka_1\text{ and }Kb=Kb_1\iff aa_1^{-1}\in K\text{ and }bb_1^{-1}\in K\implies Kab=Ka_1b_1.\]
    In this case we say that the multiplication is well-defined.
\end{definition}

\begin{lemma}\thlabel{lemma4.1}
    Let $K$ be a subgroup of $G$. The following are equivalent:
    \begin{enumerate}
        \item $K\lhd G$
        \item For all $a,b\in G$, the multiplication $Ka\cdot Kb=Kab$ is well-defined.
    \end{enumerate}
\end{lemma}

\begin{proof}\,
    \begin{itemize}
        \item[$\implies$] Let $Ka=Ka_1$ and $Kb=Kb_1$. Thus $aa_1^{-1}\in K$ and $bb_1^{-1}\in K$. To get $Kab=Ka_1b_1$ it suffices to show $ab(a_1b_1)^{-1}\in K$. Note that since $K\lhd G$, we have $aKa^{-1}\subseteq K$. Thus,
        \[ab(a_1b_1)^{-1}=abb_1^{-1}a_1^{-1}=abb_1^{-1}a^{-1}aa_1^{-1}=(a(bb_1^{-1})a^{-1})(aa_1^{-1})\in K\]
        since by assumption $K\lhd G$. It follows that $Kab=Ka_1b_1$.
        \item[$\impliedby$] It suffices to show that $aka^{-1}\in K$ for all $k\in K$ and $a\in G$. Since $Kk=K1$,
        \[Kak=Ka\cdot K1=Ka\]
        by assumption. It follows that $aka^{-1}\in K$. Thus $K\lhd G$.
    \end{itemize}
\end{proof}

\begin{proposition}[Properties of Quotient Groups]
    Let $K\lhd G$ and write $G/K=\{Ka:a\in G\}$.
    \begin{enumerate}
        \item $G/K$ is a group under the operation $Ka\cdot Kb=Kab$.
        \item The mapping $\phi:G\to G/K$ given by $\phi(a)=Ka$ is an onto homomorphism.
        \item If $[G:K]$ is finite, then $|G/K|=[G:K]$. In-particular, if $|G|$ is finite, then $|G/K|=\frac{|G|}{|K|}$.
    \end{enumerate}
\end{proposition}

\begin{proof}\,
\begin{enumerate}
    \item By \thref{lemma4.1}, the group operation is well defined and $G/K$ is closed under it. The identity element is $K=K1$ since $Ka\cdot K1=Ka=K1\cdot Ka$ for all $Ka\in G/K$. Also, since $Ka\cdot Ka^{-1}=K1=Ka^{-1}\cdot Ka$, the inverse of $Ka$ is $Ka^{-1}$. Finally, by the associativity of $G$, we have $Ka(Kb\cdot Kc)=(Ka\cdot Kb)Kc$. It follows that $G/K$ is a group.

    \item $\phi$ is clearly onto. Also, for $a,b\in G$ we have $\phi(a)\phi(b)=Ka\cdot Kb=Kab=\phi(ab)$. Thus $\phi$ is a homomorphism.

    \item If $[G:K]$ is finite, then by the definition of the index $[G:K]$, we have $|G/K|=[G:K]$. Also, if $|G|$ is finite, by Lagrange's theorem (\thref{lagrangesthm})
    \[|G/K|=[G:K]=\frac{|G|}{|K|}.\]
\end{enumerate}
\end{proof}

\begin{definition}[Quotient Groups]
    Let $K\lhd G$. The group $G/K$ of all cosets of $K$ in $G$ is called the \ul{quotient group} of $G$ by $K$. Also, the mapping $\phi:G\to G/K$ given by $\phi(a)=Ka$ is called the \ul{coset map} of $K$ in $G$.
\end{definition}

\subsection{Isomorphism Theorems}

\begin{definition}[Kernals and Images]
    Let $\alpha:G\to H$ be a group homomorphism. The \ul{kernal of $\alpha$} is
    \[\ker(\alpha):=\{g\in G:\alpha(g)=I_H\}\subseteq G\]
    and the \ul{image of $\alpha$} is
    \[\im(\alpha)=\alpha(G)=\{\alpha(g):g\in G\}\subseteq G.\]
\end{definition}

\begin{proposition}
    Let $\alpha:G\to H$ be a group homomorphism.
    \begin{enumerate}
        \item $\im(\alpha)$ is a subgroup of $H$.
        \item $\ker(\alpha)$ is a normal subgroup of $G$.
    \end{enumerate}
\end{proposition}

\begin{proof}\,
\begin{enumerate}
    \item Note that $1_H=\alpha(1_G)\in\alpha(G)$ and for $h_1=\alpha(g_1)$, $h_2=\alpha(g_2)$ in $\alpha(G)$, we have
    \[h_1h_1=\alpha(g_1)\alpha(g_2)=\alpha(g_1g_2)\in\alpha(G).\]
    
    Also, by theorem 3.1, $\alpha(g)^{-1}=\alpha(g^{-1})\in\alpha(G)$. By the subgroup test (\thref{subgrouptest}) $\alpha(G)$ is a subgroup of $H$.

    \item For $\ker(\alpha)$, note that we have $\alpha(1_G)=1_H$. Also, if $k_1,k_2\in\ker(\alpha)$, then
    \[\alpha(k_1k_2)=\alpha(k_1)\alpha(k_2)=1\dot1=1\]
    and $\alpha(k_1^{-1})=\alpha(k_1)^{-1}=1^{-1}=1$. By the subgroup test (\thref{subgrouptest}), $\ker(\alpha)$ is a subgroup of $G$.

    Now note that if $g\in G$ and $k\in\ker(\alpha)$, then
    \[\alpha(gkg^{-1})=\alpha(g)\alpha(k)\alpha(g^{-1})=\alpha(g)\alpha(g)^{-1}=1\]
    Thus $g(\ker(\alpha))g^{-1}\subseteq\ker(\alpha)$ and by the normality test (\thref{normalitytest}) we have $\ker(\alpha)\lhd G$.
\end{enumerate}
\end{proof}

\begin{theorem}[First Isomorphism Theorem]\thlabel{firstisothm}
    Let $\alpha:G\to H$ be a group homomorphism. We have
    \[G/\ker(\alpha)\iso\im(\alpha).\]
\end{theorem}

\begin{proof}
    Let $K=\ker(\alpha)$. Since $K\lhd G$, $G/K$ is a group. Define the group map $\ol{\alpha}:G/K\to\im(\alpha)$ by:
    \[\ol{\alpha}(kg)=\alpha(g)\]
    for all $kg\in G/K$.
    
    Note that:
    \[kg=kg_1\iff gg_1^{-1}\in K\iff\alpha(gg^{-1})=1\iff\alpha(g)=\alpha(g_1)\]
    Thus $\ol{\alpha}$ is well-defined and 1 to 1. Also, $\ol{\alpha}$ is clearly onto. It follows that $\ol{\alpha}$ is a group homomorphism. Thus $\im(\alpha)\iso G/\ker(\alpha)$.
\end{proof}

\begin{proposition}
    Let $\alpha:G\to H$ be a group homomorphism and $K=\ker(\alpha)$. Then $\alpha$ factors uniquely as $\alpha=\ol{\alpha}\circ\phi$, where $\phi:G\to G/K$ is the coset map and $\ol{\alpha}:G/K\to H$ is defined by $\ol{\alpha}(kg)=\alpha(g)$. Note that $\phi$ is onto and 1 to 1.
\end{proposition}

\begin{note}
    By uniquely we mean that $\ol{\alpha}$ is the only homomorphism $G/K\to H$ satisfying $\ol{\alpha}\circ\phi=\alpha$.
\end{note}

\begin{proof}
    The proof is left as an exercise.
\end{proof}

\begin{proposition}
    If $G$ is a cyclic group then either $G\iso\Z_n$ (if $G$ is finite) or $G\iso\Z$ (if $G$ is infinite).
\end{proposition}

\begin{proof}
    Let $G$ be a cyclic group and consider the map $\alpha:(\Z,+)\to G$ defined by $\alpha(k)=g^k$ for some $g\in G$ with $g\neq1$. Note that $\alpha$ is a group homomorphism and since $G$ is cyclic $\alpha$ is onto. Also note that $\ker(\alpha)=\{k\in\Z:g^k=1\}$.

    If $o(g)=\infty$ then by \thref{prop2.9} $\ker(\alpha)=\{0\}$ and so by the first isomorphism theorem (\thref{firstisothm}) $G\iso\Z/\{0\}\iso\Z$.

    On the other hand, if $o(g)=n$ then by \thref{prop2.8} $\ker(\alpha)=n\Z$ and so by the first isomorphism theorem (\thref{firstisothm}) $G\iso\Z/n\Z\iso\Z_n$.
\end{proof}

\begin{theorem}[Second Isomorphism Theorem]\thlabel{secondisothm}
    Let $H$ and $K$ be subgroups of a group $G$ with $K\lhd G$. Then $HK$ is a subgroup of $G$, $K\lhd HK$, $H\cap K\lhd H$ and $$\frac{HK}{K}\iso\frac{H}{H\cap K}.$$
\end{theorem}

\begin{proof}
    Since $K\lhd G$, $HK$ is a subgroup, $HK=KH$ and $K\lhd HK$. Consider the map $\alpha:H\to\frac{HK}{K}$ given by
    \[\alpha(h)=Kh.\]
    Then $\alpha$ is a homomorphism (exercise).
    
    Now let $x\in HK=KH$. We have $x=kh$ and so $Kx=Kkh=Kh=\alpha(h)$ so $\alpha$ is onto. By a previous theorem,
    \[\ker(\alpha)=\{h\in H:Kh=K\}=\{k\in K:Hk=H\}=H\cap K\]
    so by the first isomorphism theorem we have:
    \[\frac{H}{H\cap K}\iso\frac{HK}{K}\]
\end{proof}

\begin{theorem}[Third Isomorphism Theorem]\thlabel{thirdisothm}
    Let $K\subseteq H\subseteq G$ be groups with $K\lhd G$ and $H\lhd G$. Then $H/K\subseteq G/K$ and
    \[\frac{G/K}{H/K}\iso\frac{G}{H}.\]
\end{theorem}

\begin{proof}
    Define $\alpha:G/K\to G/H$ by $\alpha(Kg)=Hg$. To see that this map is well-defined, note that if $Kg=Kg_1$ then $gg_1^{-1}\in K\subseteq H$ and thus $Hg=Hg_1$ as well. Clearly $\alpha$ is onto as well.

    Next, note that
    \[\ker(\alpha)=\{Kg:Hg=H\}=\{Kg:g\in H\}=H\cap K.\]
    By the first isomorphism theorem, this means that
    \[\frac{G/K}{H/K}\iso\frac{G}{H}.\]
\end{proof}

\section{Group Actions}

\subsection{Cayley's Theorem}

\begin{theorem}[Cayley's Theorem]\thlabel{cayleysthm}
    If $G$ is a finite group with $|G|=n$ then $G$ is isomorphic to a subgroup of $S_n$.
\end{theorem}

\begin{proof}
    Let $G=\{g_1,\cdots, g_n\}$ and let $S_G$ be the permutation group of $G$. By identifying $g_i$ with $1\leq i\leq n$ we see that $S_G\iso S_n$. Thus, to prove the theorem it suffices to find a 1-1 homomorphism $\sigma:G\to S_G$ (so the kernel will be $\{1\}$ and we can apply the first isomorphism theorem).

    For $a\in G$ define $\mu_a:G\to G$ by $\mu_a(g)=ag$. Clearly $\mu$ is a 1-1 homomorphism and so by the first isomorphism theorem we have $G\iso G/\{1\}\iso\im\sigma$, as subgroup of $S_n$.
\end{proof}

\begin{theorem}[Extended Cayley's Theorem]\thlabel{extcayleysthm}
    Let $H$ be a subgroup of a group $G$ with $[G:H]=m<\infty$. If $G$ has no normal subgroups contained in $H$ except $\{1\}$, then $G$ is isomorphic to a subgroup of $S_m$.
\end{theorem}

\begin{proof}
    Let $X=\{g_1H,\cdots,g_mH\}$ be the set of left cosets of $G$ and $\lambda_a:X\to X$ be defined by
    \[\lambda_a(gH)=agH.\]
    Now let $\tau:G\to S_X\iso S_m$ be defined by
    \[\tau(a)=\lambda_a\]
    and defined $K=\ker\tau\subseteq H$. By the first isomorphism theorem we have $G/K\iso\im\tau$. Since $K\subseteq H$ and $K\lhd G$, by the assumption, $K=\{1\}$. It follows that $G\iso\im\tau$, a subgroup of $S_X\iso S_m$.
\end{proof}

\begin{corollary}
    Let $G$ be a finite group and $p$ be the smallest prime divisor of $|G|$. If $H$ is a subgroup of $G$ with $[G:H]=p$, then $H\lhd G$.
\end{corollary}

\begin{remark}
    This is a generalization of \thref{prop3.10} where we had $p=2$.
\end{remark}

\begin{proof}
    Let $X$ be the set of all left cosets of $H\subseteq G$ and consider the homomorphism $\tau:G\to S_X$ from the previous proof. Note that if $a\in\ker(\tau)$ then $\lambda_a\epsilon$. Thus
    \[H=\lambda_a(H)=aH\]
    and hence $a\in H$. Thus $\ker(\tau)\subseteq H$ and so by the first isomorphism theorem (\thref{firstisothm}) $G/\ker(\tau)$ is isomorphic so some subgroup of $S_p$. Now let
    \[k=[H:\ker(\tau)].\]
    By Lagrange's theorem (\thref{lagrangesthm}) this gives
    \[\left|\frac GK\right|=\frac{|G|}{|H|}\cdot\frac{|H|}{|K|}=pk\]
    and so $pk|p!\implies k|(p-1)!$.
    
    Since $k\big||H|$, $|H|\big||G|$ and $p$ is the smallest prime divisor of $|G|$, so the only prime divisor $k$ can have is $p$. However, $p$ cannot be a divisor of $k$ since $k|(p-1)!$. We conclude $k$ has no prime divisors and hence $k=1$. Therefore $\ker(\tau)=H$ and so $H\lhd G$.
\end{proof}

%\begin{proof}
%    Let $X$ be the set of all distinct left cosets of $H$ in $G$. We have $|X|=p$ and $S_x\iso S_p$. Let $\tau:G\to S_X\iso S_p$ the the group homomorphism from the previous proof. By the first isomorphism theorem we again have $G/K\iso\im\tau\subseteq S_p$. Thus $G/K$ is isomorphic to a subgroup of $S_p$. Also, since $K\subseteq H$, if $[H:K]=k$, then
%    \[|G/K|=\frac{|G|}{|K|}=\frac{|G||H|}{|H||K|}=pk.\]
%    By Lagrange's theorem, we have $pk|p!$ and hence $k|(p-1)!$. Since $k\big||H|$ which divides $|G|$, and $p$ is the smallest prime divisor of $|G|$, we see that every prime divisor of $k$ must be $\geq p$ unless $k=1$. However, since $k|(p-1)!$, this forces $k=1$ which implies $K=H$. Thus $H\lhd G$.
%\end{proof}

\subsection{Group Actions}

\begin{definition}[(Left) Group Actions]
    Let $G$ be a group and $X$ be a non-empty subset of $G$. A \ul{(left) group action} of $G$ on $X$ is a mapping $G\times X\to X$, denoted $(a,x)\to a\cdot x$ such that
    \begin{enumerate}
        \item $1\cdot x=x$ for all $x\in X$.
        \item $a\cdot(b\cdot x)=(ab)\cdot x$ for all $a,b\in G$ and $x\in X$.
    \end{enumerate}
    In this case we say \ul{$G$ acts on $X$}.
\end{definition}

\begin{note}
    Let $G$ be a group acting on an set $X\neq\emptyset$. For $a,b\in G$ and $x,y\in X$, by (1) and (2), we have
    \[a\cdot x=b\cdot y\iff(b^{-1}a)\cdot x=y.\]
    In particular, we have $a\cdot x=a\cdot y\iff x=y$.
\end{note}

\begin{example}
    If $G$ is a group, let $G$ act on itself (i.e. $X=G$) by
    \[a\cdot x=axa^{-1}\]
    for all $a,x\in G$. Note that this satisfies the definition of a group action. In this case we say $G$ acts on itself by \ul{conjugation}.
\end{example}

\begin{remark}
    For all $a\in G$, define $\sigma_a:X\to X$ by $\sigma_a(x)=a\cdot x$ for all $x\in X$. Then one can show (see assignment 5):
    \begin{enumerate}
        \item $\sigma_a\in S_X$
        \item The function $\theta:G\to S_X$ given by $\theta(a)=\sigma(a)$ is a group homomorphism with
        \[\ker\theta=\{a\in G:a\cdot x=x\text{ for all }x\in X\}\]
        Note that the group homomorphism $\theta$ gives an equivalent definition of a group action of $G$ on $X$. If $X=G$ with $|G|=n$ and $\ker\theta=1$, the map $\theta:G\to S_G\iso S_n$ show that $G$ is isomorphic to a subgroup of $S_n$ which is Cayley's theorem.
    \end{enumerate}
\end{remark}

\begin{definition}
    Let $G$ be a group acting on a set $X$ and $x\in X$. We denote by
    \[G\cdot x:=\{g\cdot x:g\in G\}\subseteq X\]
    the \ul{orbit} of $x$ and
    \[S(x):=\{g\in G:g\cdot x=x\}\subseteq G\]
    the \ul{stabilizer} of $x$.
\end{definition}

\begin{proposition}
    Let $G$ be a group on a set $X\neq\emptyset$ and $x\in X$. Let $G\cdot x$ and $S(x)$ be the orbit and stabilizer of $x$. Then:
    \begin{enumerate}
        \item $S(x)$ is a subgroup of $G$.
        \item There exists a bijection from $G\cdot x$ to $\{gS(x):g\in G\}$ and $|G\cdot x|=[G:S(x)]$.
    \end{enumerate}
\end{proposition}

\begin{proof}\,
\begin{enumerate}
    \item Since $1\cdot x=x$ we have $1\in S(x)$. Also, if $g,h\in S(x)$ then $$(gh)\cdot x=g\cdot (h\cdot x)=g\cdot x=x$$
    and
    \[g^{-1}\cdot x=g^{-1}\cdot(g\cdot x)=(g^{-1}g)\cdot x=1\cdot x=x\]
    Thus $gh,g^{-1}\in S(x)$. By the subgroup test (\thref{subgrouptest}, $S(x)$ is a subgroup of $G$.

    \item Write $S(x)=S$ and consider the map
    \[\phi:G\cdot x\to\{gS:g\in G\}\]
    defined by
    \[\phi(g\cdot x)=gS.\]
    Note that
    \begin{align*}
        g\cdot x=h\cdot x&\iff(h^{-1}g)\cdot x=x \\
        &\iff h^{-1}g\in S \\
        &\iff gS=hS
    \end{align*}
    Thus $\phi$ is well-defined and 1-1. Since $\phi$ is clearly onto, $\phi$ is a bijection. It follows that
    \[|G\cdot x|=|\{gS:g\in G\}|=[G:S].\]
\end{enumerate}
\end{proof}

\begin{theorem}[Orbit Decomposition Theorem]\thlabel{orbitdecompthm}
    Let $G$ be a group acting on a finite set $X\neq\emptyset$. Let
    \[X_f=\{x\in X:a\cdot x=x,\forall a\in G\}.\]
    Note that $x\in X_f$ if and only if $|G\cdot x|=1$. Let $G\cdot x_1,\cdots,G\cdot x_n$ denote the distinct non-singleton orbits (i.e. $|G\cdot x_i|>1$). Then
    \[|X|=|X_f|+\sum\limits_{i=1}^n[G:S(x_i)].\]
\end{theorem}

\begin{proof}
    Note that for $a,b\in G$ and $x,y\in X$:
    \begin{align*}
        a\cdot x=b\cdot y&\iff(b^{-1}a)\cdot x=y \\
        &\iff y\in G\cdot x \\
        &\iff G\cdot x=G\cdot y
    \end{align*}
    Thus two orbits are either disjoint or the same. It follows that the orbits from a disjoint cover of $X$. Since $x\in X_f$ if and only if $|G\cdot x|=1$, the set $G\setminus X_f$ contains all non-singleton orbits, which are disjoint. Thus by prop 5.4, we have:
    \[|X|=|X_f|+\sum\limits_{i=1}^n|G\cdot x_i|=|X_f|+\sum\limits_{i=1}^n[G:S(x_i)]\]
\end{proof}

\begin{definition}[Centralizers]
    Let $G$ be a group. For $x\in G$ the set
    \[C_G(x)=\{g\in G:gx=xg\}\]
    is called the \ul{centralizer of $x$} in $G$. Note that $Z(G)\subseteq C_G(x)$ for all $x\in G$.
\end{definition}

\begin{remark}
    If $G$ is a group acting on itself by conjugation, then $G_f=Z(G)$ and $S(x)=C_G(x)$ for all $x\in G$. In this case, the orbit $$G\cdot x=\{gxg^{-1}:g\in G\}$$ is called the \ul{conjugacy class} of $x$.
\end{remark}

\begin{corollary}[The Class Equation]
    Let $G$ be a finite group and let $$\{gx_1g^{-1}:g\in G\},\cdots,\{gx_ng^{-1}:g\in G\}$$
    denote the distinct non-singular conjugacy classes. Then:
    \[|G|=|Z(G)|+\sum\limits_{i=1}^n[G:C_G(x_i)]\]
\end{corollary}

\begin{proof}
    This follows directly from the orbit decomposition theorem.
\end{proof}

\begin{lemma}\thlabel{lemma5.5}
    Let $p$ be a prime and $m\in\N$. Let $G$ be a group of order $p^m$ acting on a finite set $X\neq\emptyset$. Let $X_f$ be defined as in the orbit decomposition theorem (\thref{orbitdecompthm}). Then $|X|\equiv|X_f|\mod{p}$.
\end{lemma}

\begin{proof}
    By the orbit decomposition theorem (\thref{orbitdecompthm}) we have
    \[|X|=|X_f|+\sum\limits_{i=1}^n[G:S(x)]\]
    with $[G:S(x_i)]>1$ for all $i$. Since $[G:S(x_i)]$ divides $|G|=p^m$ and $[G:S(x_i)]>1$, we have $p|[G:S(x_i)]$ for all $i$. It follows that $|X|=|X_f|\mod{p}$.
\end{proof}

\begin{theorem}[Cauchy's Theorem]\thlabel{cauchysthm}
    Let $p$ be a prime and $G$ a finite group. If $p\big||G|$ then $G$ contains an element of order $p$.
\end{theorem}

\begin{proof}
    Define
    \[X:=\{(a_1,\cdots,a_p):a_i\in G,a_1\cdots a_p=1\}.\]
    Notice that for any $(a_1,\cdots,a_{p-1})\in G^{p-1}$ we have
    \[(a_1,\cdots, a_p)\in X\iff a_p=(a_1\cdots a_{p-1})^{-1}=a_{p-1}^{-1}\cdots a_1^{-1}.\]
    This means that if $|G|=n$ then $|X|=n^{p-1}$. Since $p|n$ this means that $|X|\equiv0$ mod $p$. Let the group $\Z_p=(\Z_p,+)$ act on $X$ by ``cycling'' i.e. for $k\in\Z_p$,
    \[k\cdots(a_1,\cdots,a_p)=(a_{k+1},\cdots,a_p,a_1,\cdots,a_k).\]
    One can verify this action is well-defined (exercise).
    
    Now let $X_f$ be defined as in the orbit decomposition theorem (\thref{orbitdecompthm}). Then
    \[(a_1,\cdots,a_p)\in X_f\iff a_1=\cdots=a_p\text{ and }a_i^p=1.\]
    Clearly $(1,\cdots,1)\in X_f$ so $|X_f|\geq1$. Since $|\Z_p|=p$, by the previous lemma we have $|X_f|\equiv|X|$ mod $p$. Since $|X|\equiv0$ mod $p$ and $|X_f|\geq1$ it follows that $|X_f|\geq p$. It follows that there exists an element $a\neq1$ such that $a^p=1$.
\end{proof}

\section{Sylow Theorems}

\subsection{\emph{p}-Groups}

\begin{definition}[$p$-Groups]
    Let $p$ be a prime. A group in which every element has an order of a non-negative power of $p$ is called a $p$-group.
\end{definition}

\begin{note}
    By Cauchy's theorem we have that a finite group $G$ is a $p$-group if and only if $|G|$ is a power of $p$.
\end{note}

\begin{lemma}
    The center $Z(G)$ of a non-trivial finite $p$-group $G$ contains more than one element.
\end{lemma}

\begin{proof}
     Recall the class equation of $G$.
    \[|G|=|Z(G)|+\sum_i[G:C_G(x_i)]\]
    where $[G:C_G(x_i)]>1$. Since $G$ is a $p$-group, by the previous remark we have $|G|$ is a power of $p$. By \thref{lemma5.5} we have $|G|\equiv|Z(G)|$ mod $p$. It follows that $p|Z(G)$. Since $1\in Z(G)$, $|Z(G)|\geq 1$ and so $|Z(G)|\geq p$.
\end{proof}

\begin{lemma}\thlabel{lemma6.2}
    If $H$ is a $p$-subgroup of a finite group $G$, then $[N_G(H):H]\equiv[G:H]$ mod $p$.
\end{lemma}

\begin{proof}
    Let $X$ be the set of all left cosets of $H$ in $G$, hence $|X|=[G:H]$. Let $H$ act on $X$ by left multiplication. For $x\in G$ we have:
    \begin{align*}
        xH\in X_f&\iff hxH=xH,\forall h\in H \\
        &\iff x^{-1}hxH=H,\forall h\in H \\
        &\iff x^{-1}hx\in H,\forall h\in H \\
        &\iff x\in N_G(H)
    \end{align*}
    Thus $|X_f|$ is the number of cosets $xH$ with $x\in N_G(H)$, and hence $|X_f|=[N_G(H):H]$. By \thref{lemma5.5},
    \[[N_G(H):H]=|X_f|\equiv|X|=[G:H]\mod{p}.\]
\end{proof}

\begin{corollary}
    Let $H$ be a $p$-subgroup of a group $G$. If $p|[G:H]$ then $p|[N_G(H):H]$ and $N_G(H)\neq H$.
\end{corollary}

\begin{proof}
    Since $p|[G:H]$ by \thref{lemma6.2} we have $[N_G(H):H]\equiv0$ mod $p$ and so $p|[N_G(H):H]$. Since $|N_G(H)|\geq 1$ we have $[N_G(H):H]\geq p$ and so $N_G(H)\neq H$.
\end{proof}

\subsection{Sylow's Three Theorems}

\begin{theorem}[First Sylow Theorem]\thlabel{firstsylowthm}
    Let $G$ be a group with $|G|=p^nm$ where $p$ is a prime and $\gcd(p,m)=1$. Then $G$ contains a subgroup of order $p^i$ for all $1\leq i\leq n$. Moreover, for $i<n$ all subgroups of order $p^i$ are normal in some other subgroup of order $p^{i+1}$.
\end{theorem}

\begin{proof}
    We proceed by induction on $i$.
    \begin{itemize}
        \item\textbf{Base Case:} $i=1$.
    
        When $i=1$, since $p\big||G|$ by Cauchy's theorem (\thref{cauchysthm}) there exists $a\in G$ with $o(a)=p$ and hence $\langle a\rangle$ is a subgroup of order $p$.
    
        \item\textbf{Inductive Case:} Suppose the for some $1\leq i<n$, $G$ has a subgroup $H$ of order $p^i$.
    
        Since $G$ is finite, by Lagrange's theorem (\thref{lagrangesthm}) we have $[G:H]=\frac{|G|}{|H|}=p^{n-i}m$ so $p|[G:H]$. By the previous corollary this means $p|[N_G(H):H]$ and $[N_G(H):H]\geq p$. Thus by Cauchy's theorem (\thref{cauchysthm}) $N_G(H)/H$ contains a subgroup of order $p$. Such a group is of the form $H_1/H$ where $H_1$ is a subgroup of $N_G(H)$.
    
        Since $H\lhd N_G(H)$ we have $H\lhd H_1$. Finally, since $|H_1|$ is finite we have
        \[p=\left|\frac{H_1}{H}\right|=\frac{|H_1|}{|H|}=\frac{|H_1|}{p^i}\implies|H_1|=p^{i+1}.\]
    \end{itemize}
\end{proof}

\begin{definition}[Sylow $p$-Subgroups]
    A subgroup $P$ of a group $G$ is a \ul{Sylow $p$-subgroup} of $G$ if $P$ is a maximal $p$-subgroup of $G$ (i.e. if $P\subseteq H\subseteq G$ where $H$ is another $p$-subgroup of $G$ then $P=H$).
\end{definition}

\begin{proposition}[Properties of Sylow $p$-Subgroups]\thlabel{prop6.4}
    Let $G$ be a group of order $p^nm$ for some prime $p$ with $\gcd(p,m)=1$. The following are true:
    \begin{enumerate}
        \item For all $p$-subgroups $H$ of $G$, $H$ is a Sylow $p$-subgroup if and only if $|H|=p^n$.
        \item Every conjugate of a Sylow $p$-subgroup is a Sylow $p$-subgroup.
        \item If $G$ has exactly one Sylow $p$-subgroup $P$ then $P\lhd G$.
    \end{enumerate}
\end{proposition}

\begin{proof}
    This follows from the first Sylow theorem.
\end{proof}

\begin{theorem}[Second Sylow Theorem]\thlabel{secondsylowthm}
    If $H$ is a $p$-subgroup of a finite group $G$ and $P$ is any Sylow $p$-subgroup of $G$ then there exists $g\in G$ such that $H\subseteq gPg^{-1}$. In-particular, any two Sylow $p$-subgroups of $G$ are conjugate.
\end{theorem}

\begin{proof}
    Let $X$ be the set of all left cosets of $P$ in $H$, and let $H$ act on $X$ by left multiplication. \thref{lemma5.5} we have $|X|\equiv|X_f|=[G:P]$ mod $p$. Since $p\not|\,[G:P]$ we have $|X_f|\neq 0$. Thus there exists some $gP\in X_f$ for some $g\in G$. Note that:
    \begin{align*}
        gP\in X_f&\iff hgP=gP,\forall h\in H \\
        &\iff g^{-1}hgP=P,\forall h\in H \\
        &\iff g^{-1}hg\in P,\forall h\in H \\
        &\iff g^{-1}Hg\subseteq P \\
        &\iff H\subseteq gPg^{-1}
    \end{align*}
    Also, since $H$ is a Sylow $p$-subgroup $|H|=|P|=|gPg^{-1}|$ and so $H=gPg^{-1}$.
\end{proof}

\begin{theorem}[Third Sylow Theorem]\thlabel{thirdsylowthm}
    Let $G$ be a finite group and $p$ be a prime with $p\big||G|$. Then the number of Sylow $p$-subgroups of $G$ divides $|G|$ and is of the form $kp+1$ for some $k\in\N\cup\{0\}$.
\end{theorem}

\begin{proof}
    By the second Sylow theorem (\thref{secondsylowthm}), the number of Sylow $p$-subgroups of $G$ is the number of conjugates of any particular Sylow $p$-subgroup $P$. This number is $[G:N_G(P)]$ which divides $|G|$.

    Now $X$ be the set of all Sylow $p$-subgroups of $G$ and let $P$ act on $X$ by conjugation. Then $Q\in X_f\iff gQg^{-1}=Q$ for all $g\in P$. The latter condition holds if and only if $P\subseteq N_G(Q)$. Both $P$ and $Q$ are Sylow $p$-subgroups of $G$ and hence of $N_G(Q)$. Thus they are conjugate in $N_G(Q)$. Since $Q\lhd N_G(Q)$, this can only occur if $Q=P$. Thus $Q=P$ and $X_f=\{P\}$. Since $|X|\equiv|X|_f\equiv1$ mod $p$ this gives $|X|=kp+1$ for some $k\in N\cup\{0\}$.
\end{proof}

\begin{note}
    Suppose that $G$ is a group with $|G|=p^nm$ where $\gcd(p,m)=1$. Let $n_p$ be the number of Sylow $p$-subgroups of $G$. By the third Sylow theorem we see that $n_p|p^nm$ and $n_p\equiv1$ mod $p$. So we have $n_p|m$.
\end{note}

\section{Finite Abelian Groups}

\subsection{Primary Decomposition}

\begin{notation}
    Let $G$ be a group and $m\in\Z$. We define $G^{(m)}=\{g\in G:g^m=1\}$.
\end{notation}

\begin{proposition}\thlabel{prop7.1}
    Let $G$ be an abelian group. Then $G^{(m)}$ is a subgroup of $G$.
\end{proposition}

\begin{proof}
    This follows easily from the subgroup test (\thref{subgrouptest}).
\end{proof}

\begin{proposition}\thlabel{prop7.2}
    Let $G$ be a finite abelian group with $|G|=mk$ where $\gcd(m,k)=1$. Then:
    \begin{enumerate}
        \item $G\iso G^{(m)}\times G^{(k)}$
        \item $\left|G^{(m)}\right|=m$ and $\left|G^{(k)}\right|=k$.
    \end{enumerate}
\end{proposition}

\begin{proof}\,
    \begin{enumerate}
        \item Since $G$ is abelian we have $G^{(m)}\lhd G$ and $G^{(k)}\lhd G$. Also, since $\gcd(m,k)=1$, there exist $x,y\in\Z$ such that $mx+ky=1$.
        \begin{claim}
            $G^{(m)}\cup G^{(k)}=\{1\}$.
        \end{claim}
        \begin{proof}
            Suppose that $g\in G^{(m)}\cup G^{(k)}$. Then since $mx+ky=1$:
            \begin{align*}
                g&=g^{mx+ky} \\
                &=(g^m)^x(g^k)^y \\
                &=1^x1^y \\
                &=1
            \end{align*}
        \end{proof}
        \begin{claim}
            $G=G^{(m)}G^{(k)}$.
        \end{claim}
        \begin{proof}
            If $g\in G$ then
            \[1=(g^m)^k=g^{mk}=(g^k)^m.\]
            It follows that $g^k\in G^{(m)}$ and $g^m\in G^{(k)}$. Thus
            \[g=g^{mx+ky}=(g^k)^y(g^m)^x\in G^{(m)}G^{(k)}.\]
        \end{proof}
        Combining Claims 7.2.1 \& 7.2.2 gives $G\iso G^{(m)}\times G^{(k)}$.

        \item Let $|G^{(m)}|=m'$ and $|G^{(k)}|=k'$. By (1) we have $mk=|G|=m'k'$.
        \begin{claim}
            $\gcd(m,k')=1$.
        \end{claim}
        \begin{proof}
            Suppose $\gcd(m,k')\neq1$. Then there exists a prime $p$ such that $p|m$ and $p|k'$ and so by Cauchy's theorem (\thref{cauchysthm}) there exists a $g\in G^{(k)}$ such that $o(g)=p$. Since $p|m$ we have $g^m=1$ and so $g\in G^{(m)}$ as well. Thus by Claim 7.2.1 $g=1$, which contradicts $o(g)=p$.
        \end{proof}

        Note that since $m|m'k'$ and $\gcd(m,k')=1$ we have $m|m'$. A similar argument gives $k|k'$. Since $mk=m'k'$ this means $m=m'$ and $k=k'$.
    \end{enumerate}
\end{proof}

\begin{theorem}[Primary Decomposition Theorem]\thlabel{pdt}
    Let $G$ be a finite abelian group with $|G|=p_1^{n_1}\cdots p_k^{n_k}$ where $p_1,\cdots,p_k$ are distinct primes and $n_1,\cdots,n_k\in\N$. Then we have:
    \begin{enumerate}
        \item $G\iso G^{(p_1^{n_1})}\times\cdots\times G^{(p_k^{n_k})}$
        \item $|G^{(p_i^{n_i})}|=p_i^{n_i},\forall 1\leq i\leq k$
    \end{enumerate}
\end{theorem}

\begin{proof}
     This follows directly from the previous theorem.
\end{proof}

\subsection{Structure Theorem for Finite Abelian Groups}

\begin{proposition}\thlabel{prop7.4}
    If $G$ is a finite abelian $p$-group that contains only one subgroup of order $p$, then $G$ is cyclic.
\end{proposition}

\begin{proof}
    Let $y\in G$ be of maximal order.
    \begin{claim}
        $G=\langle y\rangle$.
    \end{claim}
    \begin{proof}
        Suppose $G\neq\langle y\rangle$. Then the quotient group $G/\langle y\rangle$ is a non-trivial $p$-group, which contains an element $z$ of order $p$ by Cauchy's theorem. In particular, $z\neq 1$. Consider the coset map $\pi:G\to G/\langle y\rangle$. Let $x\in G$ satisfy $\pi(x)=z$. Since $\pi(x^p)=z^p=1$ we see that $x^p\in\langle y\rangle$. Thus $x^p=y^m$ for some $m\in\Z$.

        \begin{itemize}
            \item\textbf{Case 1:} $p\nd m$.
    
            If $p\nd m$ then since $o(y)=p^r$ for some $r\in\N$ we have $o(y^r)=o(y)$. Since $y$ is of maximal order, we have
            \[o(x^p)<o(x)\leq o(y)=o(y^m)=o(x^p)\]
            which leads to a contradiction.
    
            \item\textbf{Case 2:} $p|m$.
    
            If $p|m$ then $m=pk$ for some $k\in\Z$. Thus we have $x^p=y^m=y^{pk}$. Since $G$ is abelian, we have $(xy^{-k})^p=1$. Thus $xy^{-k}$ belongs to the one and only subgroup of order $p$, say $H$. On the other hand, the cyclic group $\langle y\rangle$ contains a subgroup of order $p$, which must be the one and only $H$. Thus $xy^{-k}\in\langle y\rangle$, which implies that $x\in\langle y\rangle$. It follows that $z=\pi(x)=1$ which is a contradiction.
    
            By combining the above two cases we have $G=\langle y\rangle$.
        \end{itemize}
    \end{proof}
    Since $G=\langle y\rangle$, it is cyclic and the proof is complete.
\end{proof}

\begin{proposition}
        Let $G\neq\{1\}$ be a finite abelian $p$-group. Let $C$ be a cyclic subgroup of maximal order. Then $G$ contains a subgroup $B$ such that $G=CB$ and $C\cap B=\{1\}$. Thus by \thref{prop3.13} $G\iso C\times B$.
\end{proposition}

\begin{proof}
    We proceed by induction.

    \noindent\textbf{Base Case:} If $|G|=p$ then we take $C=G$ and $B=\{1\}$ and the result follows.
    
    \noindent\textbf{Inductive Case:} Suppose that the result holds for all abelian groups of order $p^{n-1}$ for some $n\in N$, $n\geq 2$. Consider $|G|=p^n$. We have two cases:

    \begin{itemize}
        \item\textbf{Case 1:} If $C=G$ then by taking $B=\{1\}$ the result follows.

        \item\textbf{Case 2:} If $G\neq C$ then $G$ is not cyclic. By theorem 7.4, there exist at least two subgroups of order $p$. Since $C$ is cyclic, it contains exactly one subgroup of order $p$. Thus there exists a subgroup $D$ of $G$ with $|D|=p$ and $D\subsetneq C$. Since $|D|=p$ and $D\subsetneq C$ we have $C\cap D=\{1\}$. Consider the coset map $\pi:G\to G/D$. If we consider $\pi\hspace{-3pt}\upharpoonright_{C_1}$, the restriction of $\pi$ onto $C$, then $\ker(\pi\hspace{-3pt}\upharpoonright_{C_1})=C\cap D=\{1\}$. Thus by the first isomorphism theorem, $\pi(C)=C$.

        Let $y$ be a generator of the cyclic group $C$. Since $\pi(C)\iso C$, $\pi(C)\iso\langle\pi(y)\rangle$. By the assumption on $C$, $\pi(C)$ is a cyclic subgroup of $G/D$ of maximal order. Since $|G/D|=p^{n-1}$, by the inductive hypothesis $G/D$ contains a subgroup $E$ such that $G/D=\pi(C)E$ and $\pi(C)\cap E=\{1\}$. Let $B=\pi^{-1}(E)$ i.e. $\pi(B)=E$.
        \begin{claim}
            $G=CB$
        \end{claim}
        \begin{proof}
            Note that $E$ is a subgroup containing $\{1\}$. We have $\pi^{-1}(\{1\})=D\subseteq B$. If $x\in G$, since $\pi(C)\pi(B)=\pi(C)E=G/D$, there exist $u\in C$ and $v\in B$ such that $\pi(x)=\pi(u)\pi(v)$. Since $\pi(xu^{-1}v^{-1})=1$ we have $xu^{-1}v^{-1}\in D\subseteq B$. Since $G$ is abelian, we have $x=uxu^{-1}\in CB$.
        \end{proof}
        \begin{claim}
            $C\cap B=\{1\}$
        \end{claim}
        \begin{proof}
            Let $x\in C\cap B$. Then \[\pi(x)\in\pi(C)\cap\pi(B)=\pi(C)\cap E=\{1\}.\]
            Since $\pi(x)=1\in G/D$, we have $x\in D$. Since $x\in C\cap D=\{1\}$ we see $x=1$.
        \end{proof}
        By Claims 7.5.1 and 7.5.2, the result follows by induction.
    \end{itemize}
\end{proof}

\begin{proposition}
    Let $G$ be a finite abelian $p$-group. Then $G$ is isomorphic to a direct product of cyclic groups.
\end{proposition}

\begin{proof}
    By the previous theorem there exists a cyclic group $C_1$ and a subgroup $B_1$ of $G$ such that $G\iso C_1\times B_1$. Since $|B_1|$ divides $|G|$ by Lagrange's theorem, the group $B_1$ is also a $p$-group. Thus if $B_1\neq\{1\}$, then by the previous theorem these exists groups $C_2$ and $B_2$ such that $B_1\iso C_2\times B_2$... Repeat until we get $B_k=\{1\}$. Thus
    \[G\iso C_1\times\cdots\times C_k.\]
\end{proof}

\begin{theorem}[Structure Theorem for Finite Abelian Groups]\thlabel{thm7.8}
    If $G$ is a finite abelian group then
    \[G\iso C_{p_1^{n_1}}\times\cdots\times C_{p_k^{n_k}}\]
    where each $C_{p_i^{n_i}}$ is a cyclic group of order $p_i^{n_i}$. The numbers $p_i^{n_i}$ are uniquely determined up to order, but the $p_i$ need not be unique.
\end{theorem}

\begin{note}
    If $p_1,\cdots,p_k$ are distinct primes then
    \[C_{p_1^{n_1}}\times\cdots\times C_{p_k^{n_k}}\iso C_{p_1^{n_1}\cdots p_k^{n_k}}.\]
\end{note}

\begin{proposition}[Invariant Factor Decomposition of Finite Abelian Groups]
    Let $G$ be a finite abelian group. Then
    \[G\iso C_{n_1}\times\cdots\times C_{n_r}\]
    where $n_i\in\N$, $n_1>1$ and $n_1|n_2|\cdots|n_r$.
\end{proposition}

\section{Rings}

\subsection{Basic Properties}

\begin{definition}[Rings]
    A \ul{ring} is a set $R$ equipped with two binary operations called addition and multiplication (which are denoted $+$ and $\cdot$) such that $(R,+)$ is an abelian group, $(R,\cdot)$ satisfies the closure, associativity and identity properties of a group and multiplication distributes over addition.

    More precisely, a set $R$ is a ring if for all $a,b,c\in R$:
    \begin{enumerate}
        \item$a+b\in R$
        \item$a+b=b+a$
        \item$a+(b+c)=(a+b)+c$
        \item There exists $0\in R$ such that $a+0=a=0+a$ (0 is called the \ul{zero} of $R$)
        \item There exists $-a\in R$ such that $a+(-a)=0=(-a)+a$ ($-a$ is called the \ul{negative} of $a$ in $R$ and we define $a-b:=a+(-b)$, $(-a)+b:=-a+b$)
        \item$ab:=a\cdot b\in R$
        \item$a(bc)=(ab)c$
        \item There exists $1\in R$ such that $a\cdot 1=1\cdot a$ (1 is called the \ul{unity} of $R$)
        \item$a(b+c)=ab+ac$ and $(a+b)c=ac+bc$
    \end{enumerate}
    Furthermore, if $R$ also has the property
    \begin{enumerate}
        \item[10.]$ab=ba$
    \end{enumerate}
    then it is a \ul{commutative ring}.
\end{definition}

\begin{note}
    Since $(R,+)$ forms a group, it follows that $-(-a)=a$.
\end{note}

\begin{example}
    $\Z,\Z_n,\Q,\R$ and $\C$ are all commutative rings, while $M_n(\R)$ is a non-commutative ring (for $n\geq 2$).
\end{example}

\begin{definition}[Repeated Multiplication and Addition]
    Given a ring $R$ and $n\in\Z$, we define $n$-times repeated addition as:
    \[na=\begin{cases}\underbrace{a+\cdots+a}_{n\text{ times}},&n>0\\0,&n=0\\\underbrace{-a-\cdots-a}_{n\text{ times}},&n<0\end{cases}\] 
    Similarly, for $n\in\N\cup\{0\}$ we define $n$-times repeated multiplication as:
    \[a^n=\begin{cases}\underbrace{a\cdots a}_{n\text{ times}},&n\in\N\\1,&n=0\end{cases}\]
    Furthermore, if there exists $a^{-1}\in R$ such that $a^{-1}a=1=aa^{-1}$ then for $n\in\N\cup\{0\}$ we also define
    \[a^{-n}=(a^{-1})^n.\]
\end{definition}

\begin{note}
    Be careful of the difference between $na$ for $n\in\Z$ and $ba$ for $b\in R$. One is repeated ring addition and the other is ring multiplication. We often write $0_R,1_R\in R$ to distinguish the identity and unity of $R$ with those of $\Z$.
\end{note}

\begin{proposition}[Algebraic Identities of Rings]\thlabel{prop8.1}
    Let $R$ be a ring, $r,s\in R$ and $n,m\in\Z$.
    \begin{enumerate}
        \item$na+ma=(n+m)a$
        \item$n(ma)=(nm)a$
        \item$n(a+b)=na+nb$
        \item$0_Rr=0_R=r0_R$
        \item$(-r)s=r(-s)=-(rs)$
        \item$(-r)(-s)=rs$
        \item$(mr)(ns)=(mn)(rs)$
    \end{enumerate}
\end{proposition}

\begin{proof}
    1,2 and 3 follow from the corresponding group identities of $(R,+)$. The rest are left as an exercise.
\end{proof}

\begin{definition}[Trivial Rings]
    A ring $R$ is a \ul{trivial ring} if $|R|=1$.
\end{definition}

\begin{remark}
    In a trivial ring we have $0_R=1_R$.
\end{remark}

\begin{definition}[Direct Products of Rings]
    Let $R_1,\cdots,R_n$ be rings. Their \ul{direct product} is the set $R_1\times\cdots\times R_n$ and we define addition and multiplication on this set (using the addition and multiplication functions of $R_1,\cdots,R_n$) as
    \[(r_1,\cdots,r_n)+(s_1,\cdots,s_n)=(r_1+s_1,\cdots,r_n+s_n)\]
    and
    \[(r_1,\cdots,r_n)\cdots(s_1,\cdots,s_n)=(r_1s_1,\cdots,r_ns_n).\]
\end{definition}

\begin{note}
    One can easily show that the set $R_1\times\cdots\times R_n$ forms a ring when equipped with the addition and multiplication operations defined above.
\end{note}

\begin{definition}[Characteristics of Rings]
    Given a ring $R$, the \ul{characteristic} of $R$ (denoted $\ch(R)$ using the order of $1\in R$ in the additive group $(R,+)$ as
    \[\ch(R)=\begin{cases}n,&o(1)=n\text{ in }(R,+)\\0,&o(1)=\infty\text{ in }(R,+)\end{cases}\]
\end{definition}

\begin{example}
    $\Z,\Q,\R$ and $\C$ all have characteristic 0, while $\Z_n$ has characteristic $n$.
\end{example}

\begin{definition}
    For $k\in\Z$ we write $kR=0$ to mean $kr=0_R$ for all $r\in R$.
\end{definition}

\begin{note}
    By \thref{prop8.1} we have $kr=k(1_Rr)=(k1_R)r$
    so $kR=0\iff k1_RR=0$.
\end{note}

\begin{theorem}
    Let $R$ be a ring and $k\in\Z$.
    \begin{enumerate}
        \item If $\ch(R)=n\in\N$ then $kR=0\iff k|n$.
        \item If $\ch(R)=0$ then $kR=0\iff k=0$.
    \end{enumerate}
\end{theorem}

\begin{proof}
    This follows from results in chapter 2 applied to the group $(R,+)$.
\end{proof}

\subsection{Subrings}

\begin{definition}[Subrings]
    A subset $S$ of a ring $R$ is a subring if $S$ is itself a ring under $R$'s addition and multiplication operations.
\end{definition}

\begin{proposition}[The Subring Test]
    A subset $S$ of a ring $R$ is a subring if it satisfies the following properties:
    \begin{enumerate}
        \item $1_R\in S$
        \item If $s,t\in S$, then $s-t,st$ are all in $S$ (note that if this is true then $s-s=0\in S$ and $0-t=-t\in S$ as well)
    \end{enumerate}
\end{proposition}

\begin{proof}
    The properties 2,3,7 and 9 of a ring are automatically satisfied in any subset of $R$. The two requirements of the subring test clearly satisfy the remaining properties.
\end{proof}

\begin{example}
    We have a chain of subrings: $\Z\subseteq\Q\subseteq\R\subseteq\C$
\end{example}

\begin{definition}[Centers of Rings]
    Given a ring $R$ the \ul{center} of $R$ is
    \[Z(R)=\{z\in R:zr=rz,\forall r\in R\}.\]
\end{definition}

\begin{note}
    We have $1_R\in Z(R)$ and if $s,t\in Z(R)$ then for all $r\in R$,
    \[(s-t)r=sr-tr=rs-rt=r(s-r)\]
    and
    \[(st)r=s(tr)=s(rt)=(sr)t=(rs)t=r(st)\]
    so by the subring test $Z(R)$ is a subring of $R$.
\end{note}

\begin{example}
    Let $\Z[i]=\{a+bi:a,b\in\Z,i^2=-1\}$. Then one can show (exercise) that $\Z[i]$ is a subring of $\C$ called the \ul{Gaussian integers}.
\end{example}

\subsection{Ideals}

\begin{definition}[Ring Cosets and Quotients]
    Let $R$ be a ring and $A$ be a subgroup of $R$ under addition. Note that since $(R,+)$ is an abelian group, $A\lhd R$. For $r\in R$, we define the \ul{$r$-coset of $A$} as:
    \[r+A=\{r+a:a\in A\}\]
    Similarly to quotient groups, we also define:
    \[R/A=\{r+A:r\in R\}\]
\end{definition}

\begin{theorem}[Ring Coset Identities]
    Let $R$ be a ring and $A$ an additive subgroup of $R$. For all $r,s\in R$ we have:
    \begin{enumerate}
        \item $r+A=s+A\iff r-s\in A$
        \item $(r+A)+(s+A)=(r+s)+A$
        \item $0+A=A$ is the additive identity of $R/A$
        \item $-(r+A)=-r+A$ is the additive inverse of $r+A$
        \item $k(r+A)=kr+A,\forall k\in \Z$
    \end{enumerate}
\end{theorem}

\begin{proof}
    This follow from known properties of cossets and quotient groups.
\end{proof}

\begin{remark}
    Given $r,r_1,s,s_1\in R$, if
    \[r+A=r_1+A,s+A=s_1+A\implies rs+A=r_1s_1+A\]
    then
    \[(r+A)(s+A)=rs+A\]
    is a well defined multiplication operation on $R/A$ making it into a ring. This is characterized in the following proposition.
\end{remark}

\begin{proposition}
    Let $A$ be a an additive subgroup of a ring $R$. For $a\in A$ define
    \[Ra=\{ra:r\in R\}\text{ and }aR=\{ar:r\in R\}.\]
    The following are all equivalent:
    \begin{enumerate}
        \item $Ra\subseteq A$ and $aR\subseteq A$ for every $a\in A$.
        \item For $r,s\in R$, the multiplication $(r+A)(s+A)=rs+A$ is well-defined.
    \end{enumerate}
\end{proposition}

\begin{proof}\,
    \begin{itemize}
        \item[$\implies$] If $r+A=r_1+A$ and $s+A=s_1+A$ then we need to show $rs+A=r_1s_1+A$. Since $(r-r_1\in A)$ and $(s-s_1)\in A$ by (1) we have:
        \begin{align*}
            rs-r_1s_1&=rs-r_1s+r_1s-r_1s_1 \\
            &=(r-r_1)s+r_1(s-s_1)\in(r-r_1)R+R(s-s_1)\subseteq A
        \end{align*}
        So $rs+A=r_1s_1+A$.
    
        \item[$\impliedby$] Let $r\in R$ and $a\in A$. We have:
        \begin{align*}
            ra+A=(r+A)(a+A)=(r+A)(0+A)=r0+A=0+A=A
        \end{align*}
        Thus $ra\in A$ and we have $Ra\subseteq A$. Similarly, $aR\subseteq A$.
    \end{itemize}
\end{proof}

\begin{definition}[Ideal]
    An additive subgroup $A$ of a ring $R$ is an \ul{ideal} of $R$ if $Ra\subseteq A$ and $aR\subseteq A$ for all $a\in A$. Thus a subset $A$ of $R$ is an ideal if $0\in A$ and for all $a,b\in A$ and $r\in R$ we have $ra,ar\in A$ and $a-b\in A$.
\end{definition}

\begin{example}
    For all rings $R$, $\{0\}$ and $R$ are both ideals.
\end{example}

\begin{note}
    If $A$ is an ideal of a ring $R$ with $1_R\in A$, then $A=R$.
\end{note}

\begin{example}
    Let $R$ be a commutative ring and $a_1,\cdots,a_n\in R$. Consider the set $I$ generated by $a_1,\cdots,a_n$, i.e. $$I=\langle a_1,\cdots,a_r\rangle=\{r_1a_1+\cdots+r_na_n:r_i\in R\}.$$ Then one can show that $I$ is an ideal (exercise).
\end{example}

\begin{proposition}
    Let $A$ be an ideal of a ring $R$. Then the additive quotient group $R/A$ is a ring with multiplication $(r+A)(s+A)=rs+A$. The unity of this group is $1+A$.
\end{proposition}

\begin{proof}
    The proof is left as an exercise.
\end{proof}

\begin{definition}[Quotient Rings]
    Let $A$ be an ideal of a ring $R$. The ring $R/A$ is called the \ul{quotient ring} of $R$ by $A$.
\end{definition}

\begin{definition}[Principal Ideals]
    Let $R$ be a commutative ring and $A$ be an ideal of $R$. If $A=aR=Ra$ for some $a\in R$ then we say $A$ is a \ul{principal ideal generated by $a$} and we write $A=\langle a\rangle$.
\end{definition}

\begin{proposition}
    All ideals of $\Z$ are of the form $\langle n\rangle$ for some $n\in\Z$. Moreover, if $\langle n\rangle\neq\langle0\rangle$ and then $n$ is unique in $\N$.
\end{proposition}

\begin{proof}
    Let $A$ be an ideal of $\Z$. If $A=\{0\}$ then $A=\langle 0\rangle$. Otherwise, choose $a\in A$ with $a\neq 0$ and $|a|$ minimal. Clearly $\langle a\rangle\subseteq A$. To prove the other includes, let $b\in A$. By the division algorithm we have $b=qa+r$ for some $q,r\in\Z$ and $0\leq r<|a|$. If $r\neq0$, since $A$ is an ideal and $a,b\in A$ we have $r=b-qa\in A$ with $|r|<|a|$, which contradicts our choice of $a$. Thus $r=0$ and $b=qa\in\langle a\rangle$.
\end{proof}

\subsection{Ring Isomorphism Theorems}

\begin{definition}[Ring Homomorphisms]
    Let $R$ and $S$ be rings. A mapping $\theta:R\to S$ is a \ul{ring homomorphism} if for all $a,b\in R$:
    \begin{enumerate}
        \item $\theta(a+b)=\theta(a)+\theta(b)$
        \item $\theta(ab)=\theta(a)\theta(b)$
        \item $\theta(1_R)=1_S$
    \end{enumerate}
\end{definition}

\begin{remark}
    (2) does not imply (3) since $\theta(1_R)$ does not necessarily have a multiplicative inverse in $S$.
\end{remark}

\begin{proposition}[Properties of Ring Homomorphisms]\thlabel{prop8.8}
    Let $\theta:R\to S$ be a ring homomorphism and let $r\in R$.
    \begin{enumerate}
        \item $\theta(1_R)=0_S$
        \item $\theta(-r)=-\theta(r)$
        \item $\theta(kr)=k\theta(r),\forall k\in\Z$
        \item $\theta(r^n)=\theta(r)^n,\forall n\in N\cup\{0\}$
        \item If $u\in R^*$ (the set elements of $R$ that have multiplicative inverses) then $\theta(u^k)=\theta(u)^k$ for all $k\in\Z$. We call such $u$ a \ul{unit} of $R$.
    \end{enumerate}
\end{proposition}

\begin{definition}[Ring Isomorphisms]
    A ring homomorphism is a \ul{ring isomorphism} if it is bijective. If there exists a ring isomorphism from $R$ to $S$ then we say $R$ and $S$ are isomorphic and we write $R\iso S$.
\end{definition}

\begin{definition}[Ring Kernals and Images]
    Let $\theta:R\to S$ be a ring homomorphism. The \ul{kernal} of $\theta$ is
    \[\ker\theta=\{r\in R:\theta(r)=0\}\subseteq R\]
    and the \ul{image} is
    \[\im\theta=\theta(R)=\{\theta(r):r\in R\}\subseteq S.\]
\end{definition}

\begin{proposition}
    Let $\theta:R\to S$ be a ring homomorphism. Then:
    \begin{enumerate}
        \item $\im\theta$ is a subring of $S$
        \item $\ker\theta$ is an ideal of $R$
    \end{enumerate}
\end{proposition}

\begin{proof}\,
    \begin{enumerate}
        \item Since $\im\theta=\theta(R)$ is an additive subgroup of $S$, it suffices to show that $\theta(R)$ is closed under multiplication and $1_S\in\theta(R)$. Note that $1_S=\theta(1_R)\in\theta(R)$. Also, if $s_1=\theta(r_1)$ and $s_2=\theta(r_2)$ are in $\theta(R)$ then
        \[s_1s_2=\theta(r_1)\theta(r_2)=\theta(r_1r_2)\in\theta(R).\]
    
        \item Since $\ker(\theta)$ is an additive subgroup of $R$ it suffices to show that $ra,ar\in\ker\theta$ for all $r\in R$ and $a\in\ker(\theta)$. If $r\in R$ and $a\in\ker\theta$ then: $$\theta(ra)=\theta(r)\theta(a)=\theta(r)\cdot0=0$$
        Thus $ra\in\ker\theta$. Similarly, $ar\in\ker\theta$.
    \end{enumerate}
\end{proof}

\begin{theorem}[First Ring Isomorphism Theorem]\thlabel{firstringisothm}
    Let $\theta:R\to S$ be a ring homomorphism. We have $$R/\ker\theta\iso\im\theta.$$
\end{theorem}

\begin{proof}
    Let $A=\ker\theta$. Since $A$ is an ideal of $R$, $R/A$ is a ring. Define the ring map
    \[\ol\theta:R/A\to\im\theta,\quad\ol{\theta}(r+A)=\theta(r)\]
    for all $r+A\in R/A$. Note that $$r+A=s+A\iff r-a\in A\iff \theta(r-s)=0\iff\theta(r)=\theta(s).$$
    Thus $\ol{\theta}$ is well-defined and 1-1. Also $\ol{\theta}$ is clearly onto. Moreover, $\ol{\theta}$ is a ring homomorphism (exercise). Thus $\ol{\theta}$ is a ring isomorphism and $R/\ker\theta\iso\im\theta$.
\end{proof}

\begin{remark}
    If $A,B$ are subrings of a ring $R$, then $A\cap B$ is also a subring. Moreover, it is the largest subring contained in both $A$ and $B$.
\end{remark}

\begin{definition}[Sums of Ring Subsets]
    To consider the smallest subring of $R$ containing two subsets (not necessarily subrings) $A$ and $B$, we define
    \[A+B:=\{a+b:a\in A,b\in B\}.\]
\end{definition}

\begin{proposition}
    Let $R$ be a ring an let $A,B$ be subsets of $R$.
    \begin{enumerate}
        \item If $A$ and $B$ are two subrings of $R$ with $1_A=1_B=1_R$, then $A\cap B$ is a subring of $R$.
        \item If $A$ is a subring and $B$ is an ideal of $R$ then $A+B$ is a subring of $R$.
        \item If $A$ and $B$ are ideals of $R$, then $A+B$ is an ideal of $R$ as well.
    \end{enumerate}
\end{proposition}

\begin{proof}
    The proof is left as an exercise.
\end{proof}

\begin{theorem}[Second Ring Isomorphism Theorem]\thlabel{secondringisothm}
    Let $A$ be a subring and $B$ an ideal of a ring $R$. Then $A+B$ is a subring of $B$, $B$ is an ideal of $A+B$, $A\cap B$ is an ideal of $A$ and
    \[\frac{A+B}{B}\iso\frac{A}{A\cap B}.\]
\end{theorem}

\begin{proof}
    See assignment 8.
\end{proof}

\begin{theorem}[Third Ring Isomorphism Theorem]
    Let $A$ and $B$ be ideal of a ring $R$ with $A\subseteq B$. Then $B/A$ is an ideal of $R/A$ and
    \[\frac{R/A}{B/A}\iso R/B.\]
\end{theorem}

\begin{proof}
    See assignment 8.
\end{proof}

\begin{example}
    Combining the third isomorphism theorem and the fact that all ideals of $\Z$ are principal, it follows that all ideals of $\Z_n$ are principal.
\end{example}

\begin{theorem}[Chinese Remainder Theorem]\thlabel{crt}
    Let $A$ and $B$ be ideals of $R$, then:
    \begin{enumerate}
        \item If $A+B=R$ then $\frac{R}{A\cap B}\iso\frac{R}{A}\times\frac{R}{B}$.
        \item If $A+B=R$ and $A\cap B=\{0\}$ then $R\iso\frac{R}{A}\times\frac{R}{B}$.
    \end{enumerate}
\end{theorem}

\begin{proof}
    (2) is a direct consequence of (1), so it suffices to prove (1). Define $\theta:R\to\frac RA\times\frac RA$ by
    \[\theta(r)=(r+A,r+B\]
    for all $r\in R$. Then $\theta$ is a ring homomorphism (exercise).
    
    To show $\theta$ is onto, let $(s+A,t+B)\in\frac RA\times\frac RB$ with $s,t\in R$. Since $A+B=R$, there exists $a\in A$ and $b\in B$ such that $a+b=1$.
    
    Let $r=sb+ta$. Then
    \[s-r=s-sb-ta=s(1-b)-ta=sa-ta=(s-t)a\in A.\]
    Thus $s+A=r+A$. Similarly, we have $t+B=r+B$. Thus
    \[\theta(r)=(r+A,r+B)=(s+A,t+B).\]
    So $\im\theta=\frac RA\times\frac RB$. Since $\ker\theta=A\cap B$, by the first isomorphism theorem, we have $R/(A\cap B)\iso \frac RA\times\frac RB$.
\end{proof}

\begin{corollary}\,
    \begin{enumerate}
        \item If $m,n\in\N$ with $\gcd(m,n)=1$, then $\Z_{mn}\iso\Z_m\times\Z_n$.
        \item If $m,n\in\N$ with $m,n\geq 2$ and $\gcd(m,n)=1$ then $\psi(mn)=\psi(m)\psi(n)$ where $\psi(m)=|\Z_m^*|$ is the Euler $\psi$ function.
    \end{enumerate}
\end{corollary}

\begin{remark}
    Let $m,n\in\Z$ with $\gcd(m,n)=1$. For $a,b\in\Z$, by the previous corollary, for $[a]\in\Z_m$ and $[b]\in\Z_n$ there exists a unique $[c]\in\Z_{mn}$ such that $[c]=[a]$ in $\Z_m$ and $[c]=[b]$ in $\Z_n$. In other words, the simultaneous congruences $x\equiv a$ (mod $m$) and $x\equiv b$ (mod $n$) has a unique solution of the form $x\equiv c$ (mod $mn$), which is the Chinese remainder theorem from MATH135.
\end{remark}

\begin{proposition}
    If $R$ is a ring with $|R|=p$ for some prime $p$ then $R\iso\Z_p$.
\end{proposition}

\begin{proof}
    Define $\theta:\Z_p\to R$ by $\theta([k])=k1_R$. Note that since $R$ is an additive subgroup of itself and $|R|=p$, by Lagrange's theorem $o(1_R)\in\{1,p\}$. Since $1_R\neq0$ we have $o(1_R)=p$. Thus
    \[[k]=[m]\iff p|(k-m)\iff(k-m)1_R=0\iff k1_R=m1_R.\]
    So $\theta$ is well-defined and 1-1. Also $\theta$ is a ring homomorphism (exercise). Since $|\Z_p|=|R|=p$, $\theta$ is also onto. We conclude $R\iso \Z_p$.
\end{proof}

\begin{example}
    What are all the possible rings of order $|R|=p^2$ (exercise)?
\end{example}

\section{Commutative Rings}

\subsection{Integral Domains and Fields}

\begin{definition}[Units]
    Let $R$ be a ring. We say $u\in R$ is a \ul{unit} if it has a multiplicative inverse in $R$, denoted by $u^{-1}$. We have $uu^{-1}=u^{-1}u$ and we write
    \[R^*=\{u\in R:\text{$u$ is a unit}\}\]
\end{definition}

\begin{note} If $u$ is a unit in $R$ and $r,s\in R$, we have
\[ur=ur\iff r=s\iff ru=su.\]
\end{note}

\begin{remark}
    One can show that $R^*$ forms a group under multiplication called the \ul{group of units of $R$}.
\end{remark}

\begin{definition}[Division Rings]
    A non-trivial ring $R$ is a \ul{division ring} if $R^*=R\setminus\{0_R\}$.
\end{definition}

\begin{definition}[Fields]
    A \ul{field} is a commutative division ring.
\end{definition}

\begin{remark}
    Recall that in MATH135 we saw that $\Z_n$ is a field if and only if $n$ is prime.
\end{remark}

\begin{theorem}[Wedderburn's Little Theorem]
    Every finite division ring is a field.
\end{theorem}

\begin{proof}
    The proof is left as an exercise (hard).
\end{proof}

\begin{definition}[Zero Divisors]
    Let $R\neq\{0\}$ be a ring. For $0\neq a\in R$ we say that $a$ is a \ul{zero divisor} if there exists $0\neq b\in R$ such that $ab=0$.
\end{definition}

\begin{proposition}
    Given a ring $R$, the following are equivalent.
    \begin{enumerate}
        \item If $ab=0$ in $R$, then $a=0$ or $b=0$.
        \item If $ab=ac$ in $R$ and $a\neq 0$, then $b=c$.
        \item If $ba=ca$ in $R$ and $a\neq0$, then $b=c$.
    \end{enumerate}
\end{proposition}

\begin{proof}
    We prove $(1)\iff(2)$. $(1)\iff(3)$ is similar.
    \begin{itemize}
        \item[$\implies$] Let $ab=ac$ with $a\neq 0$. Then $a(b-c)=0$ by (1). Since $a\neq 0$ we have $b-c=0\implies b=c$.
        
        \item[$\impliedby$] Let $ab=0$ in $R$. We have two cases:
        \begin{enumerate}
            \item If $a=0$ then we are done.
            \item If $a\neq0$ then $ab=0=a\cdot0$. By (2) we have $b=0$.
        \end{enumerate}
    \end{itemize}
\end{proof}

\begin{definition}[Integral Domains]
    A non-trivial commutative ring is an \ul{integral domain} if it has no zero divisor (i.e. if $ab=0$ in $R$ then $a=0$ or $b=0$).
\end{definition}

\begin{proposition}\thlabel{prop9.3}
    Every field is an integral domain.
\end{proposition}

\begin{proof}
    Let $ab=0$ in a field $R$. We want to show $a=0$ or $b=0$. If $a=0$ we are done so suppose $a\neq 0$. Since $R$ is a field this means $a\in R^*$ and so
    \[b=1\cdot b=a^{-1}ab=a^{-1}\cdot0=0\]
    thus $b=0$ as desired.
\end{proof}

\begin{remark}
    \,\begin{itemize}
        \item This proof also shows that every subring of a field is an integral domain.

        \item The converse of this theorem is not true. However, we do have the following proposition which is similar.
    \end{itemize}
\end{remark}

\begin{proposition}\thlabel{prop9.4}
    Every finite integral domain is a field.
\end{proposition}

\begin{proof}
    Let $R$ be a finite integral domain and $a\in R$ with $a\neq0$. Let $\theta:R\to R$ be defined by $\theta(r)=ar$. Since $R$ is an integral domain and $a\neq0$, $\theta$ is injective. In-particular, there exists a $b\in R$ such that $\theta(b)=ab=1$. Since $R$ is finite $\theta$ is also surjective. Since $R$ is commutative we have $ab=ba=1$ so $a$ is a unit. Thus $R$ is a field.
\end{proof}

\begin{proposition}
    The characteristic of an integral domain is either zero or $p$ for some prime $p$.
\end{proposition}

\begin{proof}
    Let $R$ be an integral domain. If $\ch(R)=0$ then we are done, so assume $\ch(R)=n$ for some positive integer $n$. Note that $R\neq\{0\}$ so $n\neq 1$. For the sake of contradiction, suppose that $n$ is not prime, then $n=ab$ for some $1<a,b<n$, $a,b\in \N$. It follows that
    \[(a\cdot1)(b\cdot 1)=(ab)(1\cdot 1)=n\cdot1=0\]
    which (since $R$ is an integral domain) means that either $a\cdot 1=0$ or $b\cdot 1=0$. This contradicts $o(1)=n$.
\end{proof}

\begin{example}
    Let $R$ be an integral domain with $\ch(R)=p$ for some prime $p$. Then for all $a,b\in R$ we have:
    \[(a+b)^p=\sum\limits_{i=0}^p\binom{p}{i}a^pb^{p-i}\]
    But since $p$ is prime, $p|\binom{p}{i}$ for all $1<i<p$. Since $\ch(R)=p$ this means that
    \[(a+b)^p=a^p+b^p.\]
\end{example}

\subsection{Prime and Maximal Ideals}

\begin{definition}[Prime Ideals]
    Let $R$ be a commutative ring. An ideal $P\neq R$ is a \ul{prime ideal} if whenever $r,s\in R$ satisfy $rs\in P$, then $r\in P$ or $s\in P$.
\end{definition}

\begin{proposition}\thlabel{prop9.6}
    If $R$ is a commutative ring then an ideal $P$ of $R$ is a prime ideal if and only if $R/P$ is an integral domain.
\end{proposition}

\begin{proof}
    Since $R$ is a commutative ring, so is $R/P$. Note that
    \[R/P\neq\{0\}\iff0+P\neq1+P\iff1\not\in P\iff P\neq R.\]
    Also, for $r,s\in R$ we have:
    \begin{align*}
        P\text{ is principal}&\iff(rs\in P\implies r\in P\text{ or }s\in P) \\
        &\iff((r+P)(s+P)=0+P\implies r+P=0+P\text{ or }s+P=0+P) \\
        &\iff R/P\text{ is an integral domain}
    \end{align*}
\end{proof}

\begin{definition}[Maximal Ideals]
    Let $R$ be a commutative ring, an ideal $M\neq R$ of $R$ is a \ul{maximal ideal} if whenever $A$ is an ideal such that $M\subseteq A\subseteq R$, then $A=M$ or $A=R$.
\end{definition}

\begin{proposition}\thlabel{prop9.7}
    If $R$ is a commutative ring, then an ideal $M$ of $R$ is a maximal ideal if and only if $R/M$ is a field.
\end{proposition}

\begin{proof}
    Since $R$ is a commutative ring, so is $R/M$. Notice that:
    \[R/M\neq\{0\}\iff 0+M\neq 1+M\iff 1\not\in M\iff M\neq R\]
    Also, for $r\in R$, note that $r\not\in M$ if and only if $r+M\neq 0+M$, so:
    \begin{align*}
        M\text{ is maximal}&\iff\langle r\rangle+M=R,\forall r\not\in M \\
        &\iff1\in\langle r\rangle+M,\forall r\not\in M \\
        &\iff\forall r\not\in M,\exists s+M\in R/M\text{ s.t. }(r+M)(s+M)=1+M \\
        &\iff R/M\text{ is a field}
    \end{align*}
\end{proof}

\begin{proposition}
    Every maximal ideal of a commutative ring is a prime ideal.
\end{proposition}

\begin{proof}
    This follows directly from \thref{prop9.3,prop9.6,prop9.7} (see diagram below).
\end{proof}

\begin{remark}
    The converse of this is not true.
\end{remark}

\begin{remark}
    What we have shown is the following:
    \[\begin{tabular}{|ccc|}
        \hline$I$ maximal&$\iff$&$R/I$ field\\
        $\Downarrow$&&$\Downarrow$ \\
        $I$ prime&$\iff$&$R/I$ integral domain \\
        \hline
    \end{tabular}\]
    Furthermore, by \thref{prop9.4} all four of these statements are equivalent in the case where $|R|$ is finite.
\end{remark}

\subsection{Fields of Fractions}

Recall that every subring of a field is an integral domain. The ``converse'' also holds: Every integral domain $R$ is isomorphic to a subring of a field $F$.

\begin{note}
    Let $R$ be an integral domain and let $D=R\setminus\{0\}$. Consider the set:
    \[X=R\times D=\{(r,s):r\in R,s\in D\}\]
    We say $(r,s)\equiv(r_1,s_1)$ on $X$ if and only if $rs_1=r_1s$. One can show that this is an equivalence relation (exercise). In particular:
    \begin{enumerate}
        \item $(r,s)\equiv(r,s)$
        \item $(r,s)\equiv(r_1,s_1)\implies(r_1,s_1)\equiv(r,s)$
        \item $(r,s)\equiv(r_1,s_1)$ and $(r_1,s_1)\equiv(r_2,s_2)\implies(r,s)\equiv(r_2,s_2)$
    \end{enumerate}
\end{note}

Motivated by the case $R=\Z$, we now define the following:

\begin{definition}[Fields of Fractions]
    Let $R$ be an integral domain, $D=R\setminus\{0\}$ and $X=R\times D$ as before. We define the \ul{fraction} $\frac rs$ to be the equivalence class $[(r,s)]$ of the pair $(r,s)\in X$. Let $F$ denote the set of all such fractions we call $F$ the \ul{field of fractions of $R$}.
    \[F=\left\{\frac rs:r\in R,s\in D\right\}=\left\{\frac rs:r,s\in R,s\neq0\right\}\]
    We define addition and multiplication over $F$ by:
    \[\frac rs+\frac{r_1}{s_1}=\frac{rs_1+r_1s}{ss_1}\text{ and }\frac rs\cdot\frac{r_1}{s_1}=\frac{rr_1}{ss_1}\]
    where $ss_1,rs_1+r_1s,rr_1$ are elements of $R$. Note that $ss_1\neq0$ since $R$ is an integral domain and $s,s_1\neq0$, thus these operations are well defined.
\end{definition}

\begin{example}
    One can show (exercise) the the above operations make $F$ into a field with the zero $\frac01$, the unity $\frac11$ and the negation $-\frac{r}{s}=\frac{-r}{s}$. Moreover, if $\frac{r}{s}\neq 0\in F$ then $r\neq0$ and thus $\frac sr\in F$ as well and we have
    \[\frac rs\cdot\frac sr=\frac{rs}{sr}=\frac{rs}{rs}=\frac{1}{1}=1_F\]
    so $\left(\frac rs\right)^{-1}=\frac{s}{r}$.
\end{example}

\begin{remark}
    In addition, we have $R\iso R'$ where
    \[R'=\left\{\frac r1:r\in R\right\}\subseteq F.\]
    Thus we have the following theorem:
\end{remark}

\begin{proposition}
    Let $R$ be an integral domain and $F$ be its corresponding field of fractions. Then $R$ is isomorphic to the subring
    \[R'=\left\{\frac{r}{1}:r\in R\right\}\subseteq F.\]
\end{proposition}

\begin{proof}
    The proof is left as an exercise.
\end{proof}

\section{Polynomial Rings}

\subsection{Polynomials Over Rings}

\begin{definition}[Polynomials]
    Let $R$ be a ring and $x$ a variable. We define:
    \[R[x]=\{f(x)=a_0+a_1x+\cdots+a_mx^m:m\in\N\cup\{0\},a_i\in R\}\]
    The $f(x)\in R[x]$ are called \ul{polynomials in $x$ over $R$}. If $a_m\neq0$, we say $f(x)$ has \ul{degree $m$}. This value is denoted $\deg(f)=m$ and we call $a_m$ the \ul{leading coefficient} of $f(x)$. If the leading coefficient is $a_m=1$ then we say that $f(x)$ is \ul{monic}. If $\deg(f)=0$ then $f(x)=a_0$ is a \ul{constant polynomial}. Note that
    \[f(x)=0\iff a_1,\cdots,a_m=0\]
    and in this case we define $\deg(0)=-\infty$.
\end{definition}

\begin{proposition}
    Let $R$ be a ring and $x$ a variable.
    \begin{enumerate}
        \item $R[x]$ is a ring.
        \item $R$ is a subring of $R[x]$.
        \item If $Z=Z(R)$ is the center of $R$, then $Z(R[x])=Z[x]$.
    \end{enumerate}
\end{proposition}

\begin{proof}
    The proof is left as an exercise.
\end{proof}

\begin{proposition}\thlabel{prop10.2}
    Let $R$ be an integral domain, then:
    \begin{enumerate}
        \item $R[x]$ is an integral domain.
        \item If $f,g\in R[x]\setminus\{0\}$ then $\deg(fg)=\deg(f)+\deg(g)$.
        \item The units in $R[x]$ are $R^*$.
    \end{enumerate}
\end{proposition}

\begin{proof}
    Let $f(x)=a_0+\cdots+a_mx^m$ and $g(x)=b_0+\cdots+b_nx^n$ with $a_m,b_n\neq0$. This means that $f(x)g(x)=a_0b_0+\cdots+a_mb_nx^{m+n}$. Since $R$ is an integral domain, $a_mb_n\neq0$ and so $f(x)g(x)\neq0$ and $\deg(fg)=m+n$, which proves (1) \& (2).

    As for (3), let $u(x)$ be a unit in $R[x]$ with inverse $v(x)$. Since $u(x)v(x)=1$, (2) implies that $u(x),v(x)\in R$ and (1) implies $u(x),v(x)\neq0$. Hence $u(x),v(x)\in R\setminus\{0\}=R^*$ so $R[x]^*\subseteq R^*$. Since $R^*
    \subseteq R[x]^*$ this means $R[x]^*=R^*$.
\end{proof}

\subsection{Polynomials Over Fields}

\begin{definition}[Polynomial Division]
    Let $F$ be a field and $f(x),g(x)\in F[x]$. We say that \ul{$f(x)$ divides $g(x)$} (denoted $f(x)|g(x)$) if there exists a polynomial $q(x)\in F[x]$ such that $$f(x)q(x)=g(x).$$
\end{definition}

\begin{proposition}\thlabel{prop10.3}
    Let $F$ be a field and $f(x),g(x),h(x)\in F[x]$.
    \begin{enumerate}
        \item If $f(x)|g(x)$ and $g(x)|h(x)$ then $f(x)|h(x)$.
        \item If $f(x)|g(x)$ and $f(x)|h(x)$ then for all $u(x),v(x)\in F[x]$,
        \[f(x)|u(x)g(x)+v(x)h(x).\]
    \end{enumerate}
\end{proposition}

\begin{proof}
    The proof is left as an exercise.
\end{proof}

\begin{proposition}\thlabel{prop10.4}
    Let $F$ be a field and $f(x),g(x)$ be monic polynomials. If $f(x)|g(x)$ and $g(x)|f(x)$ then $f(x)=g(x)$.
\end{proposition}

\begin{proof}
    Since $f(x)|g(x)$ and $g(x)|f(x)$ we have $f(x)r(x)=g(x)$ and $g(x)s(x)=f(x)$ for some $r,s\in F[x]$. Thus
    \[f(x)=g(x)s(x)=f(x)r(x)s(x)\]
    and so by \thref{prop10.2} we have
    \[\deg(f)=\deg(r)+\deg(s)+\deg(f)\implies\deg(r)=\deg(s)=0.\]
    Furthermore, since $f(x)$ and $g(x)$ are both monic, $r(x)$ and $s(x)$ must also be monic. Hence $r(x)=s(x)=1$ and so $f(x)=g(x)$.
\end{proof}

\begin{remark}
    Recall that for $a,b\in\Z$, if $a|b$, $b|a$ and $a,b$ are both positive then $a=b$. Thus in a sense the monic polynomials in $F[x]$ play the same role as the positive integers in $\Z$.
\end{remark}

\begin{proposition}[The Division Algorthim for Polynomials]\thlabel{prop10.5}
    Let $F$ be a field and $f(x),g(x)\in F[x]$ with $f(x)\neq0$. Then there exist unique polynomials $q(x),r(x)\in F[x]$ with $\deg(r)<\deg(f)$ such that
    \[g(x)=q(x)f(x)+r(x).\]
\end{proposition}

\begin{proof}
    This result can be derived by induction, however the proof is long and will be omitted (exercise).
\end{proof}

\begin{proposition}
    Let $F$ be a field and $f(x),g(x)\in F[x]\setminus\{0\}$. There exists a polynomial $d(x)\in F[x]$ with the following properties:
    \begin{enumerate}
        \item $d(x)$ is monic.
        \item $d(x)|f(x)$ and $d(x)|g(x)$.
        \item If $e(x)|f(x)$ and $e(x)|g(x)$ then $e(x)|d(x)$.
        \item $d(x)=u(x)f(x)+v(x)g(x)$ for some $u(x),v(x)\in F[x]$.
    \end{enumerate}
    Note that if $d(x)$ and $d_1(x)$ both satisfy the above conditions then since they are both monic by \thref{prop10.4} $d(x)=d_1(x)$. Hence $d(x)$ is also unique.
\end{proposition}

\begin{proof}
    Consider the set
    \[X=\{u(x)f(x)+v(x)g(x):u(x),v(x)\in F[x]\}.\]
    Since $f(x)\in X$, $X$ contains a non-zero polynomial. Furthermore, since $F$ is a field we may divide by the leading coefficient and so $X$ contains a monic polynomial. Now let
    \[X'=\{h(x)\in X:h(x)\text{ is monic}\}.\]
    We have seen that $X'$ is non-empty, so pick $d(x)\in X'$ with minimal degree. Clearly this choice of $d(x)$ satisfies both (1) and (4), while (3) follows from \thref{prop10.3}. One can also prove that $d(x)$ satisfies (2) using the division algorithm for polynomials (\thref{prop10.5}) (exercise).
\end{proof}

\begin{definition}[Irreducible Polynomials]
    Let $F$ be a field. A polynomial $\ell(x)\neq0$ in $F[x]$ is \ul{irreducible} if $\deg\ell\geq 1$ and whenever $\ell(x)=\ell_1(x)\ell_2(x)$ in $F[x]$, $\deg\ell_1=0$ or $\deg\ell_2=0$.
\end{definition}

\begin{proposition}
    Let $F$ be a field and $f(x)$,$g(x)\in F[x]$. If $\ell(x)\in F[x]$ is irreducible and $\ell(x)|f(x)g(x)$, then $\ell(x)|f(x)$ or $\ell(x)|g(x)$.
\end{proposition}

\begin{proof}
    The proof is left as an exercise.
\end{proof}

\begin{theorem}[Unique Factorization Theorem]
    Let $F$ be a field and let $f(x)\in F[x]$ with $\deg f\geq 1$. Then we can write
    \[f(x)=c\ell_1(x)\cdots\ell_m(x)\]
    where $c\in F^*$ and $\ell_i(x)$ are monic irreducible polynomials. This factorization is unique up to the order of $\ell_i$.
\end{theorem}

\begin{remark}
    This theorem implies the existence of infinitely many irreducible polynomials.
\end{remark}

\begin{proof}
    The proof is left as an exercise.
\end{proof}

\begin{proposition}
    Let $F$ be a field. Then all ideal of $F[x]$ are of the form $\langle h(x)\rangle=h(x)F[x]$ for some $h(x)\in F[x]$. If $\langle h(x)\rangle\neq\{0\}$ and $h(x)$ is monic then the generator is uniquely determined.
\end{proposition}

\begin{proposition}
    Let $F$ be a field and let $h(x)\in F[x]$ with $\deg h=m\geq 1$. Then the quotient ring $R=F[x]/\langle h(x)\rangle$ is given by
    \[R=\{a_0+a_1t+\cdots+a_{m-1}t^{m-1}:a_i\in F\text{ and }h(t)=0\}\]
    in which each element of $R$ can be uniquely represented in the above form.
\end{proposition}

\begin{proposition}
    Let $F$ be a field and let $h(x)\in F[x]$ with $\deg h\geq 1$. The following are all equivalent:
    \begin{enumerate}
        \item $F[x]/\langle h(x)\rangle$ is a field.
        \item $F[x]/\langle h(x)\rangle$ is an integral domain.
        \item $h(x)$ is irreducible in $F[x]$.
    \end{enumerate}
\end{proposition}

\end{document}
